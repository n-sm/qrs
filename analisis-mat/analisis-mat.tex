\documentclass[14pt,a4paper]{extarticle}
\usepackage[utf8]{inputenc}
\usepackage[spanish]{babel}
\usepackage{amsmath}
\usepackage{amsfonts}
\usepackage{amssymb}
\usepackage{amsthm}
\usepackage{float}
\usepackage{varioref}

%\usepackage{parskip}
%\makeatletter
%\def\thm@space@setup{%
%  \thm@preskip=\parskip \thm@postskip=0.5pt
%}
\makeatother
\usepackage{mdframed}
\usepackage[left=.6cm,right=.6cm,top=.6cm,bottom=.6cm]{geometry}
%\newtheorem{theorem}{Teorema}[section]
\newmdtheoremenv{theorem}{Teorema}[section]
\newtheorem{ejs}{Ejemplo}[section]
\newmdtheoremenv{mydef}{Definici\'on}[section]

\begin{document}

\section{Logaritmos}
Veamos la siguiente \emph{regla del cambio de base}:

Para todo \( x \)  se cumple \( x^{\log_x(M)} = M \). De esto se sigue
\( a^{\log_a(b)} = b \) y \( b^{\log_b(M)} = M \). Sustituyendo
obtenemos: \(M= \big(a^{\log_a b}\big)^{\log_b M} \) que es lo mismo que
\(M= a^{(\log_a b)(\log_b M)}   \). Con lo cual, tenemos:


\[a^{\log_a M} = a^{(\log_a b)(\log_b M)} \]

De los cual se sigue \( \log_a M = (\log_a b)(\log_b M) \)  y
finalmente:

\begin{equation}
\log_b(M) = \frac{\log_a(M)}{\log_a(b)}
\end{equation}


\section{L\'imites}

\subsection{Primero, consideremos las definiciones siguientes.}
%\vspace{.5cm}
\begin{mydef}[L\'imite]
\[\emph{La f\'ormula }  \qquad \lim_{x \to a}f(x) = l \qquad \emph{ significa:}\]

\begin{equation}
(\forall \epsilon > 0) \ (\exists \delta > 0) \ \forall x \quad : \quad (0<|x-a|<
\delta)
\qquad  \to \qquad ( |f(x) - l| < \epsilon   )
\end{equation}
\end{mydef}

\begin{table}[!htbp]

\begin{align*}
&\lim_{x \to a}f(x) = l  &  \equiv \qquad  &
( \forall \epsilon > 0 \ \exists \delta > 0 \ \forall x)  \qquad   0<|x-a|< \delta  \to  |f(x) - l| <
\epsilon  & \\
\\
&\lim_{x \to \infty}f(x) = l  &  \equiv  \qquad &
(\forall \epsilon > 0 \ \exists N \ \forall x) \qquad  x > |N|  \to
 |f(x) - l| < \epsilon & \\
\\
&\lim_{x \to a}f(x) = \infty  &  \equiv \qquad  &
(\forall M > 0 \ \exists \delta > 0 \ \forall x \neq a) \qquad   0<|x-a| < \delta  \to  |f(x)| >
M  & \\
\\
&\lim_{x \to a}f(x) = 0  &  \equiv   \qquad  &
(\forall \epsilon > 0 \ \exists \delta > 0 \ \forall x) \qquad   0<|x-a| < \delta  \to  |f(x)| <
\epsilon & \\
\\
&\lim_{x \to \infty}f(x) = 0  &  \equiv \qquad  &
(\forall \epsilon > 0 \ \exists N \ \forall x) \qquad   x > N  \to
 |f(x)| < \epsilon & \\
\end{align*}
\caption{definiciones de limites}
\label{tab:limDef}
\end{table}


\subsection{Algunos teoremas.}
\begin{table}[!htbp]
{\renewcommand{\arraystretch}{1.2}
%\vspace{1.5cm}
\begin{align*}
\hline \\
& \lim_{x \to a} \Big(k + f(x)\Big) & =  \qquad & k + \lim f(x)& \\
& \lim_{x \to a}\Big(f(x) + g(x)\Big) & =   \qquad  &   \lim_{x \to a}f(x) + \lim_{x  \to a} g(x)&  \\
& \lim_{x \to a}\Big(f(x) g(x)\Big) & =   \qquad  &   \lim_{x \to a}f(x) \lim_{x  \to a} g(x)&  \\
& \lim_{x \to a}\Big(\frac{f(x)}{g(x)}\Big) & =   \qquad  &\frac{\lim_{x \to a}f(x) }{\lim_{x  \to a} g(x)} &  \\
& \lim_{x \to 0}\frac{1}{x} & =   \qquad  & \infty & \\
& \lim_{x \to a}\Big(f(x)\Big)^n & = \qquad & \Big(\lim_{x \to  a}f(x)\Big)^n & \\
& \sqrt[n]{\lim_{x \to a}f(x)} & = \qquad & \lim_{x \to
  a}\sqrt[n]{f(x)} & \\
& \lim_{x \to 0}\frac{\sin{x}}{x} & = \qquad & 1 & \\
& \lim_{x \to 0}\frac{\sin{ax}}{ax} & = \qquad & 1 & a \neq 0 \\
& \lim_{x \to a}\frac{\sin{x-a}}{x-a} & = \qquad & 1 & \\
\hline \\
\end{align*}
}
\caption{Teoremas sobre l\'imites}
\label{tab:limReg}
\end{table}

\vspace{.5cm}

\subsection{M\'as definiciones.}
\begin{mydef}[Infinitesimales del mismo orden]
Sean $f(x)$ y $g(x)$ infinitesimales, cuando $x \to a$. Luego, si la
funci\'on:
\[\lim_{x \to a}\frac{f(x)}{g(x)}\]
tiene un l\'imite finito y distinto de cero, los infinitesimales
$f(x)$ y $g(x)$ son llamados \emph{del mismo orden}.
\end{mydef}
\begin{mydef}[Infinitesimales el orden superior e inferior]
Sean $f(x)$ y $g(x)$ infinitesimales, cuando $x \to a$. Luego, si la
funci\'on:
\[\lim_{x \to a}\frac{f(x)}{g(x)}=0\qquad\text{y}\qquad\lim_{x \to a}\frac{g(x)}{f(x)}=\infty\]
entonces, la infinitesimal $f(x)$ se llama \emph{infinitesimal de
  orden superior a} $g(x)$ y \'esta \'ultima se denomina
\emph{infinitesimal de orden inferior a} $f(x)$
\end{mydef}

\begin{mydef}[Infinitesimal de orden k respecto de otra]
Sean $f(x)$ y $g(x)$ infinitesimales, cuando $x \to a$. Luego, si la
funci\'on:
\[\lim_{x \to a}\frac{f(x)}{g(x)^k}=l\neq0\neq\infty\]
entonces, la infinitesimal $f(x)$ se llama \emph{infinitesimal de
  orden k respecto de} $g(x)$
\end{mydef}


\begin{mydef}[Infinitesimales equialentes]
Sean $f(x)$ y $g(x)$ infinitesimales, cuando $x \to a$. Luego, si la
funci\'on:
\[\lim_{x \to a}\frac{f(x)}{g(x)}=1\]
entonces, las infinitesimales $f(x)$ y $g(x)$ se llaman
\emph{equivalentes}.
\end{mydef}

\section{Continuidad}

\begin{mydef}[Continuidad de una funci\'on]

Una funci\'on \( f(x) \) es \emph{cont\'inua} en el punto \( x_0 \) si
est\'a definida en la vecindad de dicho punto y

\[ \lim_{\Delta x \to 0} \Delta y = 0 \]

O, lo que es igual:


\[ \lim_{\Delta x \to 0} (f(x_0+\Delta x)-f(x_0)) = 0 \]

Y como \( \lim(a+b) = \lim a + \lim b \), tambi\'en podemos escribir:

\[ \lim_{\Delta x \to 0} f(x_0+ \Delta x) = f(x_0) \]

Lo que es igual a:

\[ \lim_{x \to x_0} f(x) = f(x_0) \]



\end{mydef}

\section{Derivadas}

\begin{mydef}[Derivable en \( a \) ]
La funci\'on \( f \) es \emph{derivable en} \( a \)  si existe
\[ \lim_{h \to 0} \frac{f(a+h)-f(a)}{d} \]
\end{mydef}

\begin{theorem}
Si f es derivable en a, entonces es cont\'inua en \( a \) .
\end{theorem}

\begin{theorem}
\[ (\log_a(x))' = \frac{1}{x}\log_a(e) \]
\end{theorem}
\begin{proof}

\[ \frac{\log_a(x+h)-\log_a(x)}{h} \quad = \quad
\log_a\Big(\frac{x+h}{x}\Big)\Big(\frac{1}{h}\Big) \quad = \quad
\log_a\Big(1+\frac{h}{x}\Big)\Big(\frac{1}{h}\Big) \quad = \quad
 \]


\[  \quad = \quad \log_a\Big(1+\frac{h}{x}\Big)\Big(\frac{x}{h}\Big)\Big(\frac{1}{x}\Big)
 \quad = \quad
 \Big(\frac{1}{x}\Big)\log_a\Big(1+\frac{h}{x}\Big)^{\Big(\frac{x}{h}\Big)} \]

Y como \( \lim_{h \to 0}(1+\frac{h}{x})^{\frac{x}{h}}  =  e \) se
deduce entones que


\[ \Big(\log_a(x)\Big)' \quad = \quad \frac{1}{x} \log_a(e) \]
\end{proof}
Corolario: \( \Big( ln(x) \Big)' = \frac{1}{x}.  \)
\begin{proof}
\( \ln(e)=1 \)
\end{proof}

\subsection{Reglas de derivaci\'on}

\begin{theorem}[Regla de la cadena]
\begin{equation}
\Big(f(g(x))\Big)' = f'(g(x))g'(x)
\end{equation}
\end{theorem}


\begin{table}[!htbp]
\begin{align*}
&  f &  &  f'  & \\
 \hline  \\
& k & & 0 & \\
&  kx &  &  k   & \\
&  ku &  &   ku'   & \\
&  vu &  &  v'u + vu'   & \\
&  x^k & &  ku^{k-1}   & \\
&  u^k  & &  ku^{k-1}u'   & \\
&  a^x  & &  a^x\ln(a)   & \\
&  a^u  & &  a^u(\ln a)u'   & \\
&  u^v  & &  vu^{v-1}u'+u^v(\ln u)v'   & \\
&  log_a(x) & &   \big(\frac{1}{x}\big)\log_a(e)  & \\
&  log_a(u) & &   \big(\frac{1}{u}\big)\log_a(e)u'  & \\
&  \frac{k}{x} & &  -\frac{k}{x^2}  & \\
&  \frac{k}{u}  & &  -\frac{ku'}{u^2}   & \\
&  \frac{v}{u}  & &  \frac{v'u-vu'}{u^2}  & \\
& \sin(x) & & \cos(x) & \\
& \cos(x) & & -\sin(x) & \\
& \tg(x) & & \frac{1}{\cos^2(x)}  & \\
& \cotg(x) & & \frac{1}{\sin^2(x)}  & \\
\hline \\
\end{align*}
\caption{reglas de derivaci\'on.  \( x \) reprsenta un variable, \( a \) y \( k \)
constantes y \( u \) y \( v \) funciones de \( x \) .
}
\label{tab:derReg}
\end{table}

%~\vref{tab:derReg}
%\vpageref[above table ][table ]{tab:derReg}

\subsection{Algunos teoremas sobre el comportamiento de las funciones}

\begin{mydef}[Punto m\'aximo]
Sea \( f \) una funci\'on y \( A \) un conjunto contenido en su
dominio. Un punto mi \( x \) de \( A \) se dice que es \textbf{punto
  m\'aximo} de \( f \) sobre \( A \)  si
\begin{equation}
(\forall y \in A) \quad f(x) \geq f(y)
\end{equation}
\end{mydef}

\begin{mydef}[Valor m\'aximo]
El n\'umero \( f(x) \) recibe el nombre de \textbf{valor m\'aximo} de
\( f \) sobre \( A \) si \( x \) es su punto m\'aximo.
\end{mydef}

\begin{theorem}
Sea \( f \) una funci\'on definida sobre \( (a,b) \). Luego, si \( x \)
es un m\'aximo (para \( f \)) sobre \( (a,b) \), y \( f \) es
derivable en \( x \) , entonces \( f'(x)=0 \) .
\end{theorem}
Es importante notar aqu\'i que el rec\'iroco de este teorema no es
cierto. Es posible que se d\'e \( f'(x)=0 \) sin que por ello \( x \)
sea un punto m\'aximo. Este es el caso de \( f'(0) \) cuando \(
f(x)=x^3 \) .

\begin{mydef}[Punto m\'aximo (m\'inimo) local]
Un punto \( x \) es un m\'aximo local de la funci\'on \( f \) sobre \(
A \) si existe alg\'un \( \delta \) tal que \( x \) es punto m\'aximo
sobre el conjunto \( A \cap (x-\delta, x+\delta) \) .
\end{mydef}

\begin{theorem}
Si \( f \) est\'a definida sobre \( (a,b) \), tiene un m\'aximo local
en \( x \) y es derivable en \( x \) , entonces \( f'(x)=0 \) .
\end{theorem}

\begin{mydef}[Punto singular y valor singular]
Se llama \textbf{punto singular} de una funci\'on \( f \)  a todo
n\'umero \( x \) tal que
\[ f'(x)=0 \]
El n\'umero \( f(x) \) recibe entonces el nombre de \textbf{valor singular}.
\end{mydef}

\begin{theorem}[Teorema de Rolle] Si \( f \) es cont\'inua sobre \( [a,b] \) y derivable sobre \( (a,b) \) , y \( f(a) = f(b) \) , entonces existe un n\'umero \( x \)  en \( (a,b) \) tal que \( f'(x)=0 \) .
\end{theorem}
\begin{proof}
Como, seg\'un suponemos, \( f(x) \) es cont\'inua en \( [a,b] \), debe tener un m\'aximo \( M \) y un m\'inimo \( m \)  en \( [a,b] \) .
Si \( M = m  \), entonces \( f(x)  \) es constante y cualquier \( x \in (a,b) \) sirve.
Supongamos que \( M > 0 \geq m \)  (\(  m > M \) no podr\'ia ser) y que \( f(c)=M \). Observemos que, seg\'un lo supuesto, debe ser \( c \neq a \)  y \( c \neq b \) .
Tenemos entonces que, para todo \( \Delta x \) , \( f(c+\Delta x) - f(c) \leq 0 \) ya sea \( \Delta x > 0 \) como \( \Delta x < 0 \) . Se sigue:
{\renewcommand{\arraystretch}{1.8} %<- modify value to suit your needs
\begin{align*}
&\frac{f(c+\Delta x)-f(c)}{\Delta x} \leq 0 & cuando  & \quad \Delta x > 0 & \\
&\frac{f(c+\Delta x)-f(c)}{\Delta x} \geq 0 & cuando & \quad \Delta x < 0 & \\
\end{align*}}
Como, seg\'un la hip\'otesis la derivada \( f'(c) \) existe, obtenemos, pasando al l\'imite las f\'ormulas anteriores que \(  f'(c) \leq 0 \)  o \(  f'(c) \geq 0 \) seg\'un sea \( \Delta x \) positiva o negativa respectivamente. De esto se sigue que
\[ f'(c)=0 \]
\end{proof}
\begin{theorem}[Teorema del valor medio]
Si \( f \) es cont\'inua en \( [a,b] \) y derivable en \( (a,b) \) , entonces existe un n\'umero \( x \)  en \( (a,b) \) tal que
\[ f'(x)=\frac{f(b)-f(a)}{b-a} \]
\end{theorem}
\begin{proof}
Definamos:
\[ h(x) \quad = \quad f(x) - \Big(\frac{f(b)-f(a)}{b-a} \Big)(x-a) \]
F\'acilmente se obtiene \( h(a)=f(a) \) y tambi\'en
\[ h(b) = f(b) - \Big(\frac{f(b)-f(a)}{b-a}\Big)(b-a) = f(a) \]
As\'i, dado que \( f(a)=f(b) \) podemos aplicar el teorema de Rolle seg\'un el cual existe alg\'un \( x \) tal que
\( h'(x)=0 \). Derivamos \( h(x) \) y obtenemos

\[ \Bigg( f(x) - \Big(\frac{f(b)-f(a)}{b-a} \Big)(x-a) \Bigg)' = f'(x) - \Big(\frac{f(b)-f(a)}{b-a} \Big) \]

De manera que \( 0 = f'(x) - \Big(\frac{f(b)-f(a)}{b-a} \Big)  \) .
\end{proof}
\subsection{M\'aximos y m\'inimos}
Para localizar el m\'aximo y el m\'inimo de \( f \) en \( [a,b] \)  deben considerarse
tres clases de puntos:

\begin{enumerate}
\item Los puntos singulares de \( f \) en \( [a,b] \) .
\item Los extremos \( a \) y \( b \) .
\item Los puntos de \( x \)  de \( [a,b] \)  tales que \( f \) no es
  derivable en \( x \) .
\end{enumerate}

Si \( x \) es un punto m\'aximo, entonces evidentemente debe ser
alguno de estos tres casos.

Si existen s\'olo unos pocos puntos singulares, y s\'olo unos pocos en
los cuales no es derivable, el procedimiento para hallar los puntos
m\'aximos es: se halla \( f(x) \) para cada \( x \) que satisface \(
f'(x)=0 \) , y \( f(x) \) para cada \( x \) en que \( f \) no es
derivable y, finalmente, \( f(a) \) y \( f(b) \). El mayor de estos
valores ser\'a el m\'aximo y el menor ser\'a el m\'inimo.

\begin{ejs}{Hallar m\'aximo y \'inimo de \( f(x)=x^3-x \)}
(falta el ejemplo)
\end{ejs}


\section{Sucesiones}

\begin{mydef}{Sucesi\'on infinita}

Una sucesi\'on infinita de n\'umeros reales es una funci\'on cuyo
dominio es \(\mathbb{N}\).
\end{mydef}
\begin{mydef}{Convergencia}

Una sucesi\'on \( \{ a_n \} \) \emph{converge hacia \( l \)} si  \(
(\forall \epsilon > 0) (\exists N) (\forall n)  :   n > N
\quad \to \quad | a_n - l | < \epsilon \) y se escribe
\[ \lim_{n \to \infty} a_n = l \]
\end{mydef}
\begin{theorem}
Sea \( f \) una funci\'on definida en un intervalo abierto que contiene
\( c \), excepto quiz\'a en \( c \) mismo, con:
\[ \lim_{x \to c} f(x) = l \]

Y sup\'ongase adem\'as que \( \{ a_n \} \) es una sucesi\'on tal que:
\begin{enumerate}
\item cada \( a_n \) pertenece al dominio de \( f \),
\item cada \( a_n \neq c \),
\item \(\lim_{n \to \infty} a_n = c\)
\end{enumerate}

Entonces, la sucesi\'on \( \{ f(a_n) \} \) satisface
\[ \lim_{n \to \infty} f(a_n) = l\]
Rec\'iprocamente, si esto se cumple para toda sucesi\'on \( \{ a_n \}
\) que datisface las condiciones anteriores, entonces se cumple que
\[ \lim_{x \to c} f(x) = l \]
\end{theorem}

\begin{proof}
Para la primer parte de la prueba designemos las hip\'otesis as\'i:

\begin{align*}
&\textbf{A1.} \qquad  \forall \epsilon > 0 \ \exists \delta > 0 \ \forall x \quad
(0<|x-c|< \delta) \quad \to \quad (|f(x) - l| < \epsilon) & \\
&\textbf{A2.} \qquad \forall \epsilon > 0 \ \exists N \ \forall n \quad  (n > N) \quad \to \quad (a_n - l| < \epsilon) & \\
\end{align*}
Seg\'un como se elija \( \epsilon \) en la segunda f\'ormula, tenemos
que \( | f(a_n) - l | < \epsilon \) (es decir, usamos \( \forall \epsilon > 0 \ \exists N \ \forall n
\quad  (n > N) \quad \to \quad (|a_n - l| < \epsilon) \)
que afirma el antecedente de \( \textbf{A1} \) ).

Con esto hemos demostrado que

\[ \lim_{n \to \infty} f( a_n )= l \]

Rec\'iprocamente, designemos ahora ls hip\'otesis as\'i:

\begin{align*}
& \textbf{B1.} \qquad
 \forall \epsilon > 0 \ \exists N \ \forall n \quad  (n > N) \quad
\to \quad ( |f(a_n) - l| < \epsilon) & \\
& \textbf{B2.} \qquad
 \forall \epsilon > 0 \ \exists N \ \forall n \quad  (n > N) \quad
\to \quad ( |a_n - c| < \epsilon) & \\
\end{align*}

Ahora supongamos -a los fines de una reducci\'onn al absurdo- que no
se cumple \( \lim_{x \to c} f( x )= l \). Esto ser\'ia decir que:

\[ \exists \epsilon > 0 \ \forall \delta > 0 \ \exists x \quad
(0<|x-c|< \delta) \quad \wedge \quad (|f(x) - l| > \epsilon) \]

As\'i, para alg\'un \( \epsilon \) se cumplir\'ia que para todo \( n
\) (que es como decir, para todo \( 1/n \) existe un n\'umero \( x_n
\) tal que

\[ \exists \epsilon > 0 \ \forall n \exists x_n \quad
(0<|x_n-c|< 1/n) \quad \wedge \quad (|f(x_n) - l| > \epsilon) \]

(es decir, hemos sustitu\'ido \( \delta \) por \( 1/n \) y \( x \) por
\( x_n \)).

Con esto, la sucesi\'on \( \{ x_n \} \) converge hacia \( c \) (usamos
ac\'a \( \epsilon = 1/n \) y \( N = n \), y aplicamos la defiici\'on
). Pero \( f( \{ a_n \} \) no converger\'ia a \( l \),
contradiciendo la hip\'otesis.
\end{proof}

En los ejemplos siguientes \( N \) se refiere al \( N \) de la definici\'on
de l\'imite.

\begin{ejs}{La sucesi\'on \( \{ \frac{1}{n}\} \)} es convergente a
  cero.

\( M \): cualquier natural que verifique la condici\'on \( \frac{1}{N}
< \epsilon\) o sea \( m > \frac{1}{\epsilon} \) . (propiedd arquimediana.

\end{ejs}
\hspace{1mm}
\begin{ejs}{La sucesi\'on \( \{ x^n \} \), con \( |x|< 1 \) converge a
  cero}

Escribimos \( |x| = \frac{1}{(1+p)} \) (para \( p \) conveniente).

Y para N tomamos cualquiera que cumpla con \( \frac{1}{pN} < \epsilon \).

\end{ejs}
\hspace{1mm}
\begin{ejs}{La sucesi\'on \( \sum_{k=0}^n x^n \) llamada \emph{serie
geom\'etrica} converge a \( \frac{1}{1-x} \)  }


\end{ejs}
\hspace{1mm}
\begin{ejs}[La sucesi\'on \( \{(-1)^n\} \) no es convergente.]


\end{ejs}
\hspace{1mm}
\begin{ejs}[\(\lim_{n \to \infty} \{ ^n\sqrt{a}  \} = 1\)]
\begin{proof}
Tenemos que encontrar un \( n \) que cumpla con \( |a^{(\frac{1}{n})}|
- 1 < \epsilon \). Se dan dos casos.
\begin{flalign*}
&\textbf{Caso 1}  \qquad 0<a<1  & \\
\end{flalign*}
En este caso \( a^{\frac{1}{n}} < 1 \) para todo \( n \in \mathbf{N}
\). Por lo tanto, tenemos que \( |a^{\frac{1}{n}}-1| = 1 - a^{\frac{1}{n}} \).
De este modo, si \( 1 - a^{\frac{1}{n}} < \epsilon \) tenemos \( a^{\frac{1}{n}} > 1 - \epsilon \),
y por esto: \[ n > \frac{1 }{\log_a(1-\epsilon)} \]

\begin{flalign*}
&\textbf{Caso 2} \qquad a > 1  & \\
\end{flalign*}

Aqu\'i tenemos que \( | a^{1/n}-1| = a^{1/n}-1 \).
Tenemos que hallar un \( n \) tal que
\( a^{\frac{1}{n}}-1 < \epsilon \), es decir que
\( a^{\frac{1}{n}} < \epsilon +1\).


Lo cual se sigue de \( \log_a(\epsilon+1)=1/x \) , ya que:
\( n > x \to 1/n < 1/x \)
y \( 1/n < 1/x \to a^{1/n} < a^{1/x} \).

Como, por definici\'on de logaritmo \( a^{1/n} = \epsilon+1 \quad \equiv \quad \log_a(\epsilon+1) = \frac{1}{n} \)
y hemos supuesto la parte derecha, tenemos:
\[ a^{1/n} < \epsilon+1 \]
\end{proof}
\end{ejs}

\subsection{Algunos l\'imites}

\begin{table}[!htbp]
{\renewcommand{\arraystretch}{1.4} %<- modify value to suit your needs
\begin{flalign*}
& \textbf{Sucesi\'on} &  \textbf{L\'imite} &  & \\
\hline \\
& \lim_{n \to \infty}\frac{1}{n} & 0 & \\
& \lim_{n \to \infty} a^n & 0 & ,\quad (|a| < 1)  \\
& \lim_{n \to \infty} \frac{n}{n+1} & 0 & & \\
& \lim_{n \to \infty} \frac{n+3}{n^3+4} & 0 & & \\
& \lim_{n \to \infty} \frac{n!}{n^n} & 0 & & \\
& \lim_{n \to \infty} \sqrt{n+1} - \sqrt{n} & 0 & & \\
& \lim_{n \to \infty} \sqrt[8]{n^2+1}-\sqrt[4]{n+1} & 0 & & \\
& \lim_{n \to \infty} \sqrt[n]{a} & 1 & & \\
& \lim_{n \to \infty} \sqrt[n]{n} & 1 & & \\
& \lim_{n \to \infty} \sqrt[n]{n^2+n} & 1 & & \\
& \lim_{n \to \infty} \sqrt[n]{a^n + b^n} & \max(a,b) & & \\
& \sum_{k=0}^n x^n & \frac{1}{1-x} & \quad (\emph{Serie geometrica})
\\
\hline \\
\end{flalign*}
}
\caption{Ejemplos de l\'imites de sucesiones}
\label{tab:sucEjs}
\end{table}
\section{As\'intotas}

\subsection{As\'intota Vertical}
Se llama As\'intota Vertical de una rama de una curva \( y = f(x) \) , a la recta paralela al eje \( y \)  que hace que la rama de dicha funci\'on tienda a infinito. Si existe alguno de estos dos l\'imites:

\[ \lim_{x \to a^-} f(x) = \pm\infty \]


\[ \lim_{x \to a^+} f(x) = \pm\infty \]

a la recta \( x = a \)  se la denomina \emph{as\'intota vertical}.

Ejemplos: logaritmo neperiano, tangente
\subsection{As\'intota horizontal}
Se llama As\'intota Horizontal de una rama de una curva \(  y = f(x) \)  a la recta paralela al eje \( x \)  que hace que la rama de dicha funci\'on tienda a infinito. Si existe el l\'imite:
\[  \lim_{x \to \pm\infty} f(x)= a , \]
siendo \( a \)  un valor finito, la recta \( y = a \)  es una as\'intota horizontal.

Ejemplos: funci\'on exponencial, tangente hiperb\'olica
\subsection{As\'intota oblicua}
La recta de ecuaci\'on \( y = mx + b ,  (m \neq 0) \)  ser\'a una as\'intota oblicua si:
\[ \lim_{x \to \pm\infty}[f(x)-(mx+b)] = 0. \]

Los valores de \( m \)  y de \( b \)  se calculan con las f\'ormulas:

\[ m = \lim_{x \to \pm\infty}{f(x) \over x} \]

\[ b = \lim_{x \to \pm\infty}{f(x)-mx} \]


\end{document}
