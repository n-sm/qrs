\documentclass[12pt,a4paper]{extarticle}
\usepackage[utf8]{inputenc}
\usepackage[spanish]{babel}
\usepackage{amsmath}
\usepackage{amsfonts}
\usepackage{amssymb}
\usepackage{amsthm}
\usepackage{float}
\usepackage{varioref}
\usepackage{currfile}
\title{}
\makeatother
\usepackage{mdframed}

\newtheorem{theorem}{Teorema}[section]
%\newmdtheoremenv{theorem}{Teorema}[section]
\newtheorem{ejs}{Ejemplo}[section]
\newtheorem{mydef}{Definici\'on}[section]
\newtheorem{corol}{Corolario}[theorem]
%\newmdtheoremenv{mydef}{Definici\'on}[section]

\newcommand{\limi}[4]{
  \lim_{#1 \to #2} #3 = #4
}

\begin{document}
\maketitle
\begin{center}\currfilename\end{center}
\section{Logaritmos}
\begin{table}[!htbp]
\caption{Propiedades de los logaritmos}
\begin{align*}
\hline\\[0.5em]
&\log_a{a} &&=&& 1 &\\[0.7em]
&\log_a{1} &&=&& 0 &\\[0.7em]
&\log_a{pq} &&=&& \log_a{p} + \log_a{q}&\\[0.7em]
&\log_a{\frac{p}{q}} &&=&& \log_a{p} - \log_a{q} &\\[0.7em]
&\log_a{p^q} &&=&& q\log_a{p} &\\[0.7em]
&\log_a{p} &&=&& \frac{\log_b{p}}{\log_b{a}}&\\[0.7em]
&a^{\log_a{x}} &&=&& x &\\[0.7em]
&log_a{a^x} &&=&& x &\\[0.7em]
%&log_a{a^x} &&=&& x &\\[0.7em]
\hline \\
\end{align*}
\label{tab:loga}
\end{table}

Adem\'as, \(p \neq q \quad \to  \log_a(p) \neq \log_a(q)\) \\

Veamos la siguiente \emph{regla del cambio de base}:

Para todo \( x \)  se cumple \( x^{\log_x(M)} = M \). De esto se sigue
\( a^{\log_a(b)} = b \) y \( b^{\log_b(M)} = M \). Sustituyendo
obtenemos: \(M= \big(a^{\log_a b}\big)^{\log_b M} \) que es lo mismo que
\(M= a^{(\log_a b)(\log_b M)}   \). Con lo cual, tenemos:


\[a^{\log_a M} = a^{(\log_a b)(\log_b M)} \]

De los cual se sigue \( \log_a M = (\log_a b)(\log_b M) \)  y
finalmente:

\begin{equation}
\log_b(M) = \frac{\log_a(M)}{\log_a(b)}
\end{equation}

\section{Funciones cuadr\'aticas}

El eje de simetr\'ia de una funcione cuadr\'atica est\'a dado por:
\[
-\frac{b}{2a} = x_v\]

El valor extremo est\'a dado por \(f(-\frac{b}{2a})\), es decir:
\[
\frac{4ac-b^2}{4a} = y_v \]

El v\'ertice de la par\'abola est\'a dado por el punto \( (x_v, y_v)
\)

Los ceros de la funci\'on se obtienen con la f\'ormula:
\[
\frac{-b \pm \sqrt{b^2-4ac}}{2a} \]

Se cumple la ecuaci\'on:
\[
ax^2+bx+c = a(x-x_v)^2+\frac{4ac-b2}{4a} \]
\section{L\'imites}

\subsection{Definiciones}
\begin{minipage}{16cm}
\textbf{Definici\'on de l\'imite:}
\vspace{1em}
\hrule
\vspace{1em}
\[\lim_{x \to a}f(x) = l \qquad \emph{ significa:}\]
\[
(\forall \epsilon > 0) \ (\exists \delta > 0) \ \forall x \quad : \quad 0<|x-a|<
\delta
\qquad  \to \qquad  |f(x) - l| < \epsilon
\]

\vspace{0.5em}
\[ \lim_{x \to a}f(x) = \infty \qquad \emph{ significa:}\]
\[ (\forall M > 0 \ \exists \delta > 0 \ \forall x \neq a) \qquad   0<|x-a| < \delta  \to  |f(x)| >
M
 \]
\vspace{0.5em}

\[ \lim_{x \to \infty}f(x) = l \emph{ significa:} \]
\[ (\forall \epsilon > 0 \ \exists N \ \forall x) \qquad  x > |N|  \to
 |f(x) - l| < \epsilon
 \]
\hrule
\vspace{1em}
\end{minipage}

\begin{table}[!htbp]
\caption{Reglas sobre l\'imites}
\begin{align*}
\hline \\
& \lim_{x \to a}k & = \qquad & k &\\[0.7em]
& \lim_{x \to a}x & = \qquad & a &\\[0.7em]
& \lim_{x \to a}\Big(k + f(x)\Big) & =  \qquad & k + \lim f(x)& \\[0.7em]
& \lim_{x \to a}\Big(f(x) + g(x)\Big) & =   \qquad  &   \lim_{x \to a}f(x) + \lim_{x  \to a} g(x)&  \\[0.7em]
& \lim_{x \to a}\Big(f(x) g(x)\Big) & =   \qquad  &   \lim_{x \to a}f(x) \lim_{x  \to a} g(x)&  \\[0.7em]
& \lim_{x \to a}\Big(\frac{f(x)}{g(x)}\Big) & =   \qquad
&\frac{\lim_{x \to a}f(x) }{\lim_{x  \to a} g(x)} &
\text{\footnotesize \(\lim_{x \to a}g(x) \neq 0\)} \\[0.7em]
& \lim_{x \to a}f(x)^{g(x)} & =   \qquad  & \lim_{x \to
                                            a}f(x)^{\lim_{x \to a}g(x)} &\\[0.7em]
& \lim_{x \to a} \log{(f(x))} & =   \qquad  & \log{\big(\lim_{x \to a} f(x)\big)} &\\[0.7em]
& \lim_{x \to a}\Big(f(x)\Big)^n & = \qquad & \Big(\lim_{x \to  a}f(x)\Big)^n & \\[0.7em]
& \sqrt[n]{\lim_{x \to a}f(x)} & = \qquad & \lim_{x \to
                                            a}\sqrt[n]{f(x)} & \\[1.1em]
& \lim_{x \to c}\frac{f(x)}{g(x)} & = \qquad
                             & \lim_{x \to  c} \frac{f'(x)}{g'(x)}  \qquad
                                 & \text{\footnotesize \(f \) y \( g\)   \ cont\'inuas,
                                   definidas en \([a,b]\).}\\[-0.4cm]
&&&& \text{\footnotesize \( c \in (a,b) ,  f(a) = f(a) = 0\) o bien
     \(\pm \infty\)}\\[-0.4em]
&&&& \text{\footnotesize (L'Hopital)}\\[1em]
& \text{ Si  } b_n, c_n \to L
                  & \quad \text{ luego } &  a_n \to L
                             &  \text{ (Sandwich) } &\\
& \text{ y } b_n \leq a_n \leq c_n&&&\\[0.7em]
\hline\\
\end{align*}
\label{tab:limReg}
\end{table}

\begin{table}[!htbp]
\caption{Algunos ejemplos de l\'imites}
{\renewcommand{\arraystretch}{1.5}
%\vspace{1.5cm}
\begin{align*}
\hline \\
& \lim_{x \to 0}\frac{1}{x} && =     && \infty  &\\[1em]
& \lim_{x \to \infty}\Big(\frac{x+1}{x}\Big)&&=  && 1 &\\[1em]
&\lim_{x \to 1}\Big(\frac{x^n-1}{x-1}\Big)
                            && =  && n & \text{por L'Hopital}\\[1em]
&\lim_{x \to \infty}\Big(\frac{x^n-1}{x-1}\Big)&& =  && \infty
                                                &\text{por L'Hopital}\\[1em]
&\lim_{x \to \infty}\Big(\frac{x}{1+x}\Big)^x&& =   && e^{-1} &\\[1em]
&\lim_{x \to 0} \sen{x} &&= && 0 &\\[1em]
&\lim_{x \to 0} \cos{x} &&= && 1 &\\[1em]
& \lim_{x \to 0}\frac{\sen{x}}{x}
                            && = && 1 & \text{por L'Hopital}\\[1em]
& \lim_{x \to 0}\frac{\sen{4x}}{x} && = && 4 & \text{por L'Hopital}\\[1em]
& \lim_{x \to 0}\frac{\sen{ax}}{ax} && = && 1 , \ a \neq 0 & \
                                                             \text{por
                                                             L'Hopital} \\[1em]
& \lim_{x \to a}\frac{\sen{x-a}}{x-a} && = && 1  &\\[1em]
& \limi{x}{0}{\ln(x) &&}{ && - \infty} &\\[1em]
& \limi{x}{\infty}{\ln(x) &&}{ && \infty} &\\[1em]
& \limi{x}{0}{\frac{\ln(x)}{x} &&}{ && 0} &\ \text{por L'Hopital}\\[1em]
\vspace{1cm}\\
\hline \\
\end{align*}
}
\label{tab:limReg}
\end{table}

\begin{table}[!htbp]
\caption{ALgunas funciones que no tienden a ning\'un l\'imite}
{\renewcommand{\arraystretch}{1.5}
%\vspace{1.5cm}
\begin{align*}
\hline \\
&\lim_{x \to \infty}\sen{x} &\\[0.7em]
&\lim_{x \to 0}\sen{\frac{1}{x}} &\\
\vspace{1cm}\\
\hline \\
\end{align*}
}
\label{tab:nolim}
\end{table}

\vspace{.5cm}

\subsection{Regla de L'Hopital}
\begin{minipage}{16cm}
Sean \(f(x)\) y \(g(x)\) tales que son
\begin{itemize}
\item funciones cont\'inuas en un entorno de \(a\),
\item con derivadas en dicho entorno,
\item siendo \(g(x) \neq 0\) cerca de \(a\),
\item con \[\lim_{x \to a} f(x) = \lim_{x \to a} g(x) = 0,\]
\item y existe el \[\lim_{x \to a} \frac{f'(x)}{g'(x)},\]
\end{itemize}
entonces: \[\lim_{x \to a} \frac{f(x)}{g(x)}= \lim_{x \to a} \frac{f'(x)}{g'(x)}\]
\end{minipage}
\section{Continuidad}

\begin{mydef}[Continuidad de una funci\'on]

Una funci\'on \( f(x) \) es \emph{cont\'inua} en el punto \( x_0 \) si
est\'a definida en la vecindad de dicho punto y

\[ \lim_{\Delta x \to 0} \Delta y = 0 \]

O, lo que es igual:


\[ \lim_{\Delta x \to 0} (f(x_0+\Delta x)-f(x_0)) = 0 \]

Y como \( \lim(a+b) = \lim a + \lim b \), tambi\'en podemos escribir:

\[ \lim_{\Delta x \to 0} f(x_0+ \Delta x) = f(x_0) \]

Lo que es igual a:

\[ \lim_{x \to x_0} f(x) = f(x_0) \]
\end{mydef}

\section{Derivadas}

\begin{mydef}[Derivable en \( a \) ]
La funci\'on \( f \) es \emph{derivable en} \( a \)  si existe
\[ \lim_{h \to 0} \frac{f(a+h)-f(a)}{d} \]
\end{mydef}

\begin{theorem}
Si f es derivable en a, entonces es cont\'inua en \( a \) .
\end{theorem}

\begin{theorem}
\[ (\log_a(x))' = \frac{1}{x}\log_a(e) \]
\end{theorem}
\begin{proof}

\[ \frac{\log_a(x+h)-\log_a(x)}{h} \quad = \quad
\log_a\Big(\frac{x+h}{x}\Big)\Big(\frac{1}{h}\Big) \quad = \quad
\log_a\Big(1+\frac{h}{x}\Big)\Big(\frac{1}{h}\Big) \quad = \quad
 \]


\[  \quad = \quad \log_a\Big(1+\frac{h}{x}\Big)\Big(\frac{x}{h}\Big)\Big(\frac{1}{x}\Big)
 \quad = \quad
 \Big(\frac{1}{x}\Big)\log_a\Big(1+\frac{h}{x}\Big)^{\Big(\frac{x}{h}\Big)} \]

Y como \( \lim_{h \to 0}(1+\frac{h}{x})^{\frac{x}{h}}  =  e \) se
deduce entones que


\[ \Big(\log_a(x)\Big)' \quad = \quad \frac{1}{x} \log_a(e) \]
\end{proof}
Corolario: \( \Big( ln(x) \Big)' = \frac{1}{x}.  \)
\begin{proof}
\( \ln(e)=1 \)
\end{proof}

\subsection{Reglas de derivaci\'on}

\begin{theorem}[Regla de la cadena]
\begin{equation}
\Big(f(g(x))\Big)' = f'(g(x))g'(x)
\end{equation}
\end{theorem}

\begin{theorem}
\[(y^u)'=y^u\Big(u'\ln y+\frac{uy'}{y}\Big)\]
\end{theorem}

\begin{theorem}[Teorema de la func\'on inversa]
\[(f^{-1})'(x) = \frac{1}{f'(f^{-1}(x))}\]
\end{theorem}

\begin{table}[!htbp]
\caption{reglas de derivaci\'on.  \( x \) reprsenta un variable, \( a \) y \( k \)
constantes y \( u \) y \( v \) funciones de \( x \) .
}
\begin{align*}
&  f &  &  f'  & \\[0.75em]
 \hline  \\
& k & & 0 & \\[0.75em]
&  kx &  &  k   & \\[0.75em]
& u+v & & u' + v' &\\[0.75em]
&  ku &  &   ku'   & \\[0.75em]
&  vu &  &  v'u + vu'   & \\[0.75em]
&  x^k & &  ku^{k-1}   & \\[0.75em]
&  u^k  & &  ku^{k-1}u'   & \\[0.75em]
&  a^x  & &  a^x\ln(a)   & \\[0.75em]
&  a^u  & &  a^u(\ln a)u'   & \\[0.75em]
&  u^v  & &  vu^{v-1}u'+u^v(\ln u)v'   & \\[0.75em]
&  y^v  & &  y^v(v'\ln y + \frac{vy'}{y})   & \\[0.75em]
&  log_a(x) & &   \Big(\frac{1}{x}\Big)\log_a(e)  & \\[0.75em]
&  log_a(u) & &   \Big(\frac{1}{u}\Big)\log_a(e)u'  & \\[0.75em]
&  \frac{k}{x} & &  -\frac{k}{x^2}  & \\[0.75em]
&  \frac{k}{u}  & &  -\frac{ku'}{u^2}   & \\[0.75em]
&  \frac{v}{u}  & &  \frac{v'u-vu'}{u^2}  & \\[0.75em]
& \sen(x) & & \cos(x) & \\[0.75em]
& \cos(x) & & -\sen(x) & \\[0.75em]
& \tg(x) & & \frac{1}{\cos^2(x)}  & \\[0.75em]
& \cotg(x) & & \frac{1}{\sen^2(x)}  & \\[0.75em]
& (f^{-1})'(x) & & \frac{1}{f'(f^{-1}(x))} &\\[0.75em]
\hline \\
\end{align*}
\label{tab:derReg}
\end{table}

%~\vref{tab:derReg}
%\vpageref[above table ][table ]{tab:derReg}

\subsection{Algunos teoremas sobre el comportamiento de las funciones}

\begin{mydef}[Punto m\'aximo]
Sea \( f \) una funci\'on y \( A \) un conjunto contenido en su
dominio. Un punto  \( x \) de \( A \) se dice que es \textbf{punto
  m\'aximo} de \( f \) sobre \( A \)  si
\begin{equation}
(\forall y \in A) \quad f(x) \geq f(y)
\end{equation}
\end{mydef}

\begin{mydef}[Valor m\'aximo]
El n\'umero \( f(x) \) recibe el nombre de \textbf{valor m\'aximo} de
\( f \) sobre \( A \) si \( x \) es su punto m\'aximo.
\end{mydef}

\begin{theorem}
Sea \( f \) una funci\'on definida sobre \( (a,b) \). Luego, si \( x \)
es un m\'aximo (para \( f \)) sobre \( (a,b) \), y \( f \) es
derivable en \( x \) , entonces \( f'(x)=0 \) .
\end{theorem}
Es importante notar aqu\'i que el rec\'iroco de este teorema no es
cierto. Es posible que se d\'e \( f'(x)=0 \) sin que por ello \( x \)
sea un punto m\'aximo. Este es el caso de \( f'(0) \) cuando \(
f(x)=x^3 \) .

\begin{mydef}[Punto m\'aximo (m\'inimo) local]
Un punto \( x \) es un m\'aximo local de la funci\'on \( f \) sobre \(
A \) si existe alg\'un \( \delta \) tal que \( x \) es punto m\'aximo
sobre el conjunto \( A \cap (x-\delta, x+\delta) \) .
\end{mydef}

\begin{theorem}
Si \( f \) est\'a definida sobre \( (a,b) \), tiene un m\'aximo local
en \( x \) y es derivable en \( x \) , entonces \( f'(x)=0 \) .
\end{theorem}

\begin{mydef}[Punto singular y valor singular]
Se llama \textbf{punto singular} de una funci\'on \( f \)  a todo
n\'umero \( x \) tal que
\[ f'(x)=0 \]
El n\'umero \( f(x) \) recibe entonces el nombre de \textbf{valor singular}.
\end{mydef}

% teorema de Rolle
\begin{theorem}[Teorema de Rolle] Si \( f \) es cont\'inua sobre \( [a,b] \) y derivable sobre \( (a,b) \) , y \( f(a) = f(b) \) , entonces existe un n\'umero \( x \)  en \( (a,b) \) tal que \( f'(x)=0 \) .
\end{theorem}
\begin{proof}
Como, seg\'un suponemos, \( f(x) \) es cont\'inua en \( [a,b] \), debe tener un m\'aximo \( M \) y un m\'inimo \( m \)  en \( [a,b] \) .
Si \( M = m  \), entonces \( f(x)  \) es constante y cualquier \( x \in (a,b) \) sirve.
Supongamos que \( M > 0 \geq m \)  (\(  m > M \) no podr\'ia ser) y que \( f(c)=M \). Observemos que, seg\'un lo supuesto, debe ser \( c \neq a \)  y \( c \neq b \) .
Tenemos entonces que, para todo \( \Delta x \) , \( f(c+\Delta x) - f(c) \leq 0 \) ya sea \( \Delta x > 0 \) como \( \Delta x < 0 \) . Se sigue:
{\renewcommand{\arraystretch}{1.8} %<- modify value to suit your needs
\begin{align*}
&\frac{f(c+\Delta x)-f(c)}{\Delta x} \leq 0 & cuando  & \quad \Delta x > 0 & \\
&\frac{f(c+\Delta x)-f(c)}{\Delta x} \geq 0 & cuando & \quad \Delta x < 0 & \\
\end{align*}}
Como, seg\'un la hip\'otesis la derivada \( f'(c) \) existe, obtenemos, pasando al l\'imite las f\'ormulas anteriores que \(  f'(c) \leq 0 \)  o \(  f'(c) \geq 0 \) seg\'un sea \( \Delta x \) positiva o negativa respectivamente. De esto se sigue que
\[ f'(c)=0 \]
\end{proof}
\begin{theorem}[Teorema del valor medio]
Si \( f \) es cont\'inua en \( [a,b] \) y derivable en \( (a,b) \) , entonces existe un n\'umero \( x \)  en \( (a,b) \) tal que
\[ f'(x)=\frac{f(b)-f(a)}{b-a} \]
\end{theorem}
\begin{proof}
Definamos:
\[ h(x) \quad = \quad f(x) - \Big(\frac{f(b)-f(a)}{b-a} \Big)(x-a) \]
F\'acilmente se obtiene \( h(a)=f(a) \) y tambi\'en
\[ h(b) = f(b) - \Big(\frac{f(b)-f(a)}{b-a}\Big)(b-a) = f(a) \]
As\'i, dado que \( f(a)=f(b) \) podemos aplicar el teorema de Rolle seg\'un el cual existe alg\'un \( x \) tal que
\( h'(x)=0 \). Derivamos \( h(x) \) y obtenemos

\[ \Bigg( f(x) - \Big(\frac{f(b)-f(a)}{b-a} \Big)(x-a) \Bigg)' = f'(x) - \Big(\frac{f(b)-f(a)}{b-a} \Big) \]

De manera que \( 0 = f'(x) - \Big(\frac{f(b)-f(a)}{b-a} \Big)  \) .
\end{proof}

\begin{corol}
Si se define \(f\) sobre un intervalo y \(f'(x)=0\) para todo \(x\)
del itnervalo, entonces \(f\) es constante en el intervalo.
\end{corol}
\begin{proof}
Sean \(a\) y \(b\) dos puntos distintos del inervalo. Luego debe haber alg\'un
\(x\) tal que

\[f'(x)=\frac{f(b)-f(a)}{b-a}\]

Pero como \(f'(x)=0\) para todo \(x\) del intervalo:

\[0=\frac{f(b)-f(a)}{b-a}\]

Luego \(f(b) = f(a) \).

\end{proof}
\begin{corol}
Si \(f\) y \(g\) esta\'an definidas en el mismo intervalo y \(f'(x) = g'(x)\)
para todo \(x\) del intervalo, entonces existe alg\'un \(c\) tal que \(f + g =
c \).
\end{corol}

\begin{corol}
Si \(f'(x) > 0\) para todo \(x\) de un intervalo, entonces \(f\) es
creciente en el intervalo; si \(f'(x) < 0\) para todo \(x\) de un intervalo, entonces \(f\) es
decreciente en el intervalo.
\end{corol}
\begin{theorem}[Teorema de Cauchy]
Sean \(f(x)\) y \(g(x)\) funciones cont\'inuas y derivables dentrode
\([a,b]\). Si \(g'(x)\) no adquiere el valor cero en \([a,b]\), luego:

\[ \exists c \in (a,b) \quad . \quad \frac{f(b)-f(a)}{g(b)-g(a)} = \frac{f'(c)}{g'(c)} \]
\end{theorem}


\begin{tabular}{|c|c|}
  \hline
  & \\
  Fermat & Sea \(f\) definida en \((a,b)\) con extremo local en \(x_0
           \in (a,b)\)\\[0.5em]
  & Si \(f\) es derivable en \(x_0\) entonces \(f'(x_0) = 0\)\\[2.5em]
  Rolle &
          Si \( f \) es cont\'inua sobre \( [a,b] \) y derivable sobre
          \( (a,b) \) , y \( f(a) = f(b) \), entonces  \\[0.5em]
        & \(\exists  x \in (a,b) \quad f'(x)=0 \) . \\[2.5em]

  Valor medio &
                Si \( f \) es cont\'inua en \( [a,b] \) y derivable en
                \( (a,b) \) , entonces\\[0.5em]
  (Lagrange) & \(\exists  x \in (a,b) \quad f'(x)=\frac{f(b)-f(a)}{b-a} \) \\[2.5em]
  Cauchy
        & Sean \(f(x)\) y \(g(x)\) funciones cont\'inuas y derivables
          dentro de \([a,b]\).\\[0.5em]
 &  Si \(g'(x)\) no adquiere el valor cero en \([a,b]\), luego:\\[0.5em]
 & \( \exists c \in (a,b) \quad . \quad \frac{f(b)-f(a)}{g(b)-g(a)} = \frac{f'(c)}{g'(c)} \) \\[1em]
\hline
\end{tabular}
\subsection{M\'aximos y m\'inimos}
Para localizar el m\'aximo y el m\'inimo de \( f \) en \( [a,b] \)  deben considerarse
tres clases de puntos:

\begin{enumerate}
\item Los puntos singulares de \( f \) en \( [a,b] \) .
\item Los extremos \( a \) y \( b \) .
\item Los puntos de \( x \)  de \( [a,b] \)  tales que \( f \) no es
  derivable en \( x \) .
\end{enumerate}

Si \( x \) es un punto m\'aximo, entonces evidentemente debe ser
alguno de estos tres casos.

Si existen s\'olo unos pocos puntos singulares, y s\'olo unos pocos en
los cuales no es derivable, el procedimiento para hallar los puntos
m\'aximos es: se halla \( f(x) \) para cada \( x \) que satisface \(
f'(x)=0 \) , y \( f(x) \) para cada \( x \) en que \( f \) no es
derivable y, finalmente, \( f(a) \) y \( f(b) \). El mayor de estos
valores ser\'a el m\'aximo y el menor ser\'a el m\'inimo.

\begin{ejs}{Hallar m\'aximo y \'inimo de \( f(x)=x^3-x \)}
(falta el ejemplo)
\end{ejs}

\begin{theorem}[Criterio de la segunda derivada]
Sea \(f'(a)=0\) y que la segunda derivada de \(f\) existe en un
intervalo abierto que contiene a \(a\). Luego, \\
\\
Si \( f ''(x) < 0 \) , entonces \( f \)  tiene un máximo relativo en \( (x, f(x)) \) .\\
Si \( f ''(x) > 0 \) , entonces \( f \)  tiene un mínimo relativo en \( (x, f(x)) \) .\\
Si \( f ''(x) = 0 \) , entonces el criterio falla. Esto es, f quizás
tenga un máximo relativo en \( x \) , un mínimo relativo en \( (x,
f(x)) \)  o ninguno de los dos. Tomar como ejemplo la función \( f(x)
= x^3 \) . En tales casos, se puede utilizar el criterio de la primera
derivada o el criterio de la tercera derivada.\\
\\
Nota: la derivada segunda es la derivada de la derivada, o sea:

\[ f''(x)= (f'(x))' \]

\end{theorem}
\begin{table}[!htbp]
\caption{M\'inimos y m\'aximos}
\begin{center}
\begin{tabular}{ l c}
\hline & \\[0.25em]
\(f(c)\)  es valor m\'inimo o m\'aximo &   \(f'(x) = 0 \) \\[1em]
\( f'(x) > 0 \quad \forall x \in \mathbf{I} \) &  \(f\)  es creciente en \textbf{I}\\[1em]
\( f'(x) < 0 \quad \forall x \in \mathbf{I} \) &  \(f\) es decreciente
en \textbf{I}\\[1em]
\(f\) es c\'oncava en \(\mathbf{I}\) & \(f'\) es creciente
\(\mathbf{I}\)\\[1em]
\(f\) es convexa en \(\mathbf{I}\) & \(f'\) es decreciente
\(\mathbf{I}\)\\[1em]
\(f''(x) > 0  \quad \forall x \in \mathbf{I}\) & \(f\) es c\'oncava en
\textbf{I}\\[1em]
\(f''(x) < 0  \quad \forall x \in \mathbf{I}\) & \(f\) es convexa en
\textbf{I}\\[1em]
\hline\\
\end{tabular}
\end{center}
\label{tab:minmax}
\end{table}
\vspace{0.5em}

\begin{minipage}{16cm}
\begin{center}
\emph{Criterio de la primera derivada:}
\end{center}
\hrule
\vspace{1cm}
Si \(f'(x)>0 \ \forall \ x \in (a,c)\) y \(f'(x)<0 \ \forall \ x \in
(c,b)\) \textbf{entonces} \(f(c)\) es valor m\'aximo relativo \\[1em]
Si \(f'(x)<0 \ \forall \ x \in (a,c)\) y \(f'(x)>0 \ \forall \ x \in
(c,b)\) \textbf{entonces} \(f(c)\) es valor m\'inimo relativo \\[1em]
Si \(f'(x)\) tiene el mismo valor a ambos lados de \(c\) entocnes \(c\) no es
valor extremo\\
\vspace{0.5em}
\end{minipage}
\vspace{2em}

\begin{minipage}{16cm}
\begin{center}
\emph{Criterio de la segunda derivada:}
\end{center}
\hrule
\vspace{1em}
Sup\'ongase que \(f'\) y \(f''\) existen en todo punto de un intervalo
abierto (a,b) que contiene a \(c\) y sup\'ongase que \(f'(c)=0\).\\[1em]
\(f''(c) < 0\) \textbf{entonces} \(f(c)\) es m\'aximo relativo. \\[1em]
\(f''(c) > 0\) \textbf{entonces} \(f(c)\) es m\'inimo relativo .\\[1em]
\end{minipage}
\begin{theorem}[Teorema de Bolzano]
Sea \(f\) cont\'inua en \([a,b]\), con \(f(a) < 0\) (o \(f(a) > 0\)) y
\(f(a) > 0 \) (o \(f(a) < 0\)). Luego
\[\exists x \in (a,b) : f(x) = 0\]
\end{theorem}
\begin{theorem}[Teorema de los valores intermedios]
Sea \(f\) cont\'inua en \([a,b]\) Luego:
\[\forall u : f(a) < u < f(b) \quad \to \quad \exists x  f(x) = u \]
\end{theorem}
Como corolario tenemos que:
\[\not \exists x f(x) = 0 \quad \to \quad  \neg (f(a) < 0 < f(b))\]
Con lo cual, si tenemos que en un intervalo una función no se anula y
toma alg\'un valor positivo, entonces es positiva en ese intervalo.

\subsection{Construcci\'on de curvas}
Para la construcci\'on de un gr\'afico de una funci\'on nos valdremos
esencialmente de la siguiente informaci\'on:
\begin{itemize}
  \item Dominio \(f\).
  \item Intervalos de crecimiento y decrecimiento, y extremos locales.
  \item Intervalos de concavidad positiva y negativa, y puntos de inflexi\'on.
  \item Existencia de as\'intotas verticales, horizontales y obl\'icuas.
\end{itemize}
\section{Sucesiones}

\begin{mydef}{Sucesi\'on infinita}

Una sucesi\'on infinita de n\'umeros reales es una funci\'on cuyo
dominio es \(\mathbb{N}\).
\end{mydef}
\begin{mydef}{Convergencia}

Una sucesi\'on \( \{ a_n \} \) \emph{converge hacia \( l \)} si  \(
(\forall \epsilon > 0) (\exists N) (\forall n)  :   n > N
\quad \to \quad | a_n - l | < \epsilon \) y se escribe
\[ \lim_{n \to \infty} a_n = l \]
\end{mydef}
\begin{theorem}
Sea \( f \) una funci\'on definida en un intervalo abierto que contiene
\( c \), excepto quiz\'a en \( c \) mismo, con:
\[ \lim_{x \to c} f(x) = l \]

Y sup\'ongase adem\'as que \( \{ a_n \} \) es una sucesi\'on tal que:
\begin{enumerate}
\item cada \( a_n \) pertenece al dominio de \( f \),
\item cada \( a_n \neq c \),
\item \(\lim_{n \to \infty} a_n = c\)
\end{enumerate}

Entonces, la sucesi\'on \( \{ f(a_n) \} \) satisface
\[ \lim_{n \to \infty} f(a_n) = l\]
Rec\'iprocamente, si esto se cumple para toda sucesi\'on \( \{ a_n \}
\) que datisface las condiciones anteriores, entonces se cumple que
\[ \lim_{x \to c} f(x) = l \]
\end{theorem}


En los ejemplos siguientes \( N \) se refiere al \( N \) de la definici\'on
de l\'imite.

\begin{ejs}{La sucesi\'on \( \{ x^n \} \), con \( |x|< 1 \) converge a
  cero}

Escribimos \( |x| = \frac{1}{(1+p)} \) (para \( p \) conveniente).

Y para N tomamos cualquiera que cumpla con \( \frac{1}{pN} < \epsilon \).

\end{ejs}
\hspace{1mm}
\begin{ejs}{La sucesi\'on \( \sum_{k=0}^n x^n \) llamada \emph{serie
geom\'etrica} converge a \( \frac{1}{1-x} \)  }


\end{ejs}
\hspace{1mm}
\begin{ejs}[La sucesi\'on \( \{(-1)^n\} \) no es convergente.]


\end{ejs}
\hspace{1mm}
\begin{ejs}[\(\lim_{n \to \infty} \{ ^n\sqrt{a}  \} = 1\)]
\begin{proof}
Tenemos que encontrar un \( n \) que cumpla con \( |a^{(\frac{1}{n})}|
- 1 < \epsilon \). Se dan dos casos.
\begin{flalign*}
&\textbf{Caso 1}  \qquad 0<a<1  & \\
\end{flalign*}
En este caso \( a^{\frac{1}{n}} < 1 \) para todo \( n \in \mathbf{N}
\). Por lo tanto, tenemos que \( |a^{\frac{1}{n}}-1| = 1 - a^{\frac{1}{n}} \).
De este modo, si \( 1 - a^{\frac{1}{n}} < \epsilon \) tenemos \( a^{\frac{1}{n}} > 1 - \epsilon \),
y por esto: \[ n > \frac{1 }{\log_a(1-\epsilon)} \]

\begin{flalign*}
&\textbf{Caso 2} \qquad a > 1  & \\
\end{flalign*}

Aqu\'i tenemos que \( | a^{1/n}-1| = a^{1/n}-1 \).
Tenemos que hallar un \( n \) tal que
\( a^{\frac{1}{n}}-1 < \epsilon \), es decir que
\( a^{\frac{1}{n}} < \epsilon +1\).


Lo cual se sigue de \( \log_a(\epsilon+1)=1/x \) , ya que:
\( n > x \to 1/n < 1/x \)
y \( 1/n < 1/x \to a^{1/n} < a^{1/x} \).

Como, por definici\'on de logaritmo \( a^{1/n} = \epsilon+1 \quad \equiv \quad \log_a(\epsilon+1) = \frac{1}{n} \)
y hemos supuesto la parte derecha, tenemos:
\[ a^{1/n} < \epsilon+1 \]
\end{proof}
\end{ejs}
\subsection{Criterio de Cauchy o de la ra\'iz en\'esima}
Si \[ \lim_{n \to \infty} \sqrt[n]{a_n} = L \] entonces vale que: \\
\[ 0 \leq L \leq 1  \quad \rightarrow \quad \lim_{n \to \infty} a_n =
0 \]

\[ L \geq 1  \quad \rightarrow \quad \lim_{n \to \infty} a_n =
\infty \]

\[ L = 1  \quad \rightarrow \quad \text{Indeterminado} \]


\subsection{Criterio de D'Alembert o del cociente}
Si \[ \lim_{n \to \infty} |\frac{a_{n + 1}}{a_n}| = L \] entonces vale que: \\
\[ 0 \leq L \leq 1  \quad \rightarrow \quad \lim_{n \to \infty} a_n =
0 \]

\[ L \geq 1  \quad \rightarrow \quad \lim_{n \to \infty} a_n =
\infty \]

\[ L = 1  \quad \rightarrow \quad \text{Indeterminado} \]

\subsection{Algunos l\'imites}

\begin{table}[!htbp]
\caption{Ejemplos de l\'imites de sucesiones}
{\renewcommand{\arraystretch}{1.4} %<- modify value to suit your needs
\begin{flalign*}
& \textbf{Sucesi\'on} &  \textbf{L\'imite} &  & \\
\hline \\
& \lim_{n \to \infty}\frac{1}{n} & 0 & \\
& \lim_{n \to \infty} a^n & 0 & ,\quad (|a| < 1)  \\
& \lim_{n \to \infty} \frac{n}{n+1} & 1 & & \\
& \lim_{n \to \infty} \frac{n+3}{n^3+4} & 0 & & \\
& \lim_{n \to \infty} \frac{n!}{n^n} & 0 & & \\
& \lim_{n \to \infty} \sqrt{n+1} - \sqrt{n} & 0 & & \\
& \lim_{n \to \infty} \sqrt[8]{n^2+1}-\sqrt[4]{n+1} & 0 & & \\
& \lim_{n \to \infty} \sqrt[n]{a} & 1 & , a > 1 & \\
& \lim_{n \to \infty} \sqrt[n]{n} & 1 & & \\
& \lim_{n \to \infty} \sqrt[n]{n^2+n} & 1 & & \\
& \lim_{n \to \infty} \sqrt[n]{a^n + b^n} & \max(a,b) & & \\
& \sum_{k=0}^n x^n & \frac{1}{1-x} & \quad (\emph{Serie geometrica})
\\
& \lim_{n \to \infty}\Big(1+\frac{1}{n}\Big)^n & e & & \\
& \lim_{n \to \infty}\Big(1+\frac{z}{n}\Big)^{n} & e^z & & \\
& \lim_{n \to \infty}\Big(1+\frac{1}{zn}\Big)^{n} & e^{\frac{1}{z}} & & \\
& \lim_{n \to \infty}\Big(1-\frac{z}{n}\Big)^{n} & e^{-z} & & \\
\vspace{1cm}\\
\hline \\
\end{flalign*}
}
\label{tab:sucEjs}
\end{table}

\section{As\'intotas}

\subsection{As\'intota Vertical}
Se llama As\'intota Vertical de una rama de una curva \( y = f(x) \) , a la recta paralela al eje \( y \)  que hace que la rama de dicha funci\'on tienda a infinito. Si existe alguno de estos dos l\'imites:

\[ \lim_{x \to a^-} f(x) = \pm\infty \]


\[ \lim_{x \to a^+} f(x) = \pm\infty \]

a la recta \( x = a \)  se la denomina \emph{as\'intota vertical}.

Ejemplos: logaritmo neperiano, tangente
\subsection{As\'intota horizontal}
Se llama As\'intota Horizontal de una rama de una curva \(  y = f(x) \)  a la recta paralela al eje \( x \)  que hace que la rama de dicha funci\'on tienda a infinito. Si existe el l\'imite:
\[  \lim_{x \to \pm\infty} f(x)= a , \]
siendo \( a \)  un valor finito, la recta \( y = a \)  es una as\'intota horizontal.

Ejemplos: funci\'on exponencial, tangente hiperb\'olica
\subsection{As\'intota oblicua}
La recta de ecuaci\'on \( y = mx + b ,  (m \neq 0) \)  ser\'a una as\'intota oblicua si:
\[ \lim_{x \to \pm\infty}[f(x)-(mx+b)] = 0. \]

Los valores de \( m \)  y de \( b \)  se calculan con las f\'ormulas:

\[ m = \lim_{x \to \pm\infty}{f(x) \over x} \]

\[ b = \lim_{x \to \pm\infty}{f(x)-mx} \]

\end{document}
