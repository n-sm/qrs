\documentclass[12pt,a4paper]{extarticle}
\usepackage[utf8]{inputenc}
\usepackage[spanish]{babel}
\usepackage{amsmath}
\usepackage{amsfonts}
\usepackage{amssymb}
\usepackage{amsthm}
\usepackage{float}
\usepackage{varioref}
\usepackage{currfile}
\title{}
\makeatother
\usepackage{mdframed}
\usepackage[left=1.6cm,right=1.6cm,top=1.6cm,bottom=1.6cm]{geometry}
\newtheorem{theorem}{Teorema}[section]
%\newmdtheoremenv{theorem}{Teorema}[section]
\newtheorem{ejs}{Ejemplo}[section]
\newtheorem{mydef}{Definici\'on}[section]
\newtheorem{corol}{Corolario}[theorem]
%\newmdtheoremenv{mydef}{Definici\'on}[section]

\newcommand{\limi}[4]{
  \lim_{#1 \to #2} #3 = #4
}

\begin{document}
\maketitle
\begin{center}\currfilename\end{center}

\section{F\'ormula de Taylor}

Teorema: si \(a_k\) son los coeficientes de un polinomio,
\[
P_{n, a, f} = \sum_{k=0}^n \frac{f^{(n)} (x-a)^k}{k!}
\]

\[
\exists c \qquad R_{n,a} =  \frac{f^{(n+1)}(c)}{(n+1)!}(x-a)^{n+1}
\]

\section{Integrales}


\begin{mydef}[Partici\'on]
Sea \( a < b \) . Recibe el nombre de \emph{partici\'on} del itnervalo
\( [a,b] \) toda colecci\'on finita de puntos \( [a,b] \), de los
cuales uno es \( a \) y el otro es \( b \) .
\end{mydef}

\begin{mydef}[Suma Inferior (Superior)]
Sea \( f \) acotada sobre \( [a,b] \) y \( P = {t_0, \dots, t_n} \)
una partici\'on de \( [a,b] \) . Sea adem\'as
\[ m_i= \inf\{f(x):t_{i-1} \leq x \leq t_i\} \]
\[ M_i= \sup\{f(x):t_{i-1} \leq x \leq t_i\} \]
Luego, se llama \emph{suma inferior} de \( f \)  para \( P \) a :
\[ L(f,P)= \sum_{i=1}^n m_i(t_i-t_{i-1}) \]
Y se llama \emph{suma superio} de \( f \)  para \( P \) a :
\[ U(f,P)= \sum_{i=1}^n M_i(t_i-t_{i-1}) \]
\end{mydef}

\begin{theorem}
Sean \( P_1 \) y \( P_2 \) particiones de \( [a,b] \). Entonces:
\[ L(f,P_1)\leq U(f,P_2) \]
\end{theorem}

\begin{mydef}[Integrable]
Una funci\'on \( f \) acotada sobre \( [a,b] \)  es \emph{integrable}
sobre \( [a,b] \) si

\[
\sup\{L(f,P): P \text{ es partici\'on de } [a,b] \} =
\inf\{U(f,P) : P \text{ es partici\'on de } [a,b] \}
\]
\end{mydef}

\begin{theorem}
Si \( f \) est\'a acotada sobre \( [a,b] \) , entonces \( f \) es
integrable sobre \( [a,b] \) si y s\'olo si para todo \( \epsilon >0
\) existe una partici\'on \( P \) de \( [a,b] \) tal que

\[U(f,P)-L(f,P) < \epsilon  \]
\end{theorem}

\begin{theorem}
Si \(f\) es cont\'inua en \([a,b]\), entonces \(f\) es integrable en \([a,b]\).
\end{theorem}

\begin{theorem}
Sea \( f \) integrable sobre \( [a,b] \) y
\[ \forall x \in [a,b] : m \leq f(X) \leq M \]

Luego
\[ m(b-a) \leq \int_a^b f \leq M(b-a) \]
\end{theorem}

\begin{theorem}
Se \( f \) es integrable sobre \( [a,b] \)  y \( F \) est\'a definida
sobre \( [a,b] \) por
\[ F(x) = \int_a^x f \]
entonces \( F \) es cont\'inua sobre \( [a,b] \)

\end{theorem}

\begin{table}[!htbp]
\caption{Reglas de integrales}
{\renewcommand{\arraystretch}{1.6} %<- modify value to suit your needs
\begin{flalign*}
\hline \\
& \int_b^a f(x) \ dx & = & &-\int_a^b f(x) \ dx \\
& \int_a^a f(x) \ dx & = & & 0 \\
& \int_a^b kf(x) \ dx &  = & & k\int_a^b f(x) \ dx \\
& \int_a^b f(x)+g(x) \ dx &  = & & \int_a^b f(x) \ dx + \int_a^b g(x)
\ dx \\
& \int_a^b f(x) \ dx + \int_b^c f(x) \ dx &  = & & \int_a^c f(x) \ dx
\\
\vspace{1cm}\\
\hline \\
\end{flalign*}
}
\end{table}

\begin{table}[!htbp]
\caption{Tabla de integrales inmediatas}
{\renewcommand{\arraystretch}{1.9} %<- modify value to suit your needs
\begin{flalign*}
\hline\\
& \int a \ dx && ax &\\
& \int x^n \ dx && \frac{x^{n+1}}{n+1} \qquad , n \neq -1 & \\
& \int a^x \ dx && \frac{a^x}{\ln(a)} &\\
& \int \frac{1}{x} \ dx \quad  = \quad \int \frac{dx}{x} && \ln(x) &\\
& \int \frac{dx}{2\sqrt{x}}  && \sqrt{x} &\\
& \int e^x \ dx && e^x &\\
& \int \sen(x) \ dx && -\cos(x) &\\
& \int \cos(x) \ dx && \sen(x) &\\
& \int \frac{1}{1+x^2} \ dx && \arctan(x) &\\
& \int \frac{1}{\sqrt{1+x^2}} \ dx && \arcsin(x) &\\
\vspace{1cm}\\
\hline \\
\end{flalign*}
}
\end{table}

\section{Series}
\begin{theorem}
Si una serie \(s_n\) converge y \(u_n\) se obtiene en base a ella
suprimiendo un n\'umero finito de t\'erminos en ella, luego \(u_n\) converge.
\end{theorem}
\begin{theorem}
Si \(s_n\) converge y suma \(s\)  y \(u_n\) se obtiene multimplicando cada t\'ermino
de \(s\) por \(c\), luego \(u_n\) converge en y suma \(cs\).
\end{theorem}
\begin{theorem}
Si las series \(s_n\) y \(u_n\) convergen y suman \(s\) y \(u\)
respectivamente, tambi\'en lo hace la serie que resulta de sumar cada
en\'esimo t\'ermino de una con el de la otra y su suma es \(s + u\).
\end{theorem}
\vspace{1em}

\begin{theorem}[Condici\'on necesaria para la convergencia]
Si una serie converge, entonce si en'esimo t\'ermino tiende a cero
cuando \(n \to 0\).
\end{theorem}
\begin{corol}
Si el en\'esimo t\'ermino de una serie \(s_n\) no tiende a cero,
entonces \(s_n\) no converge.
\end{corol}
\vspace{1em}
\begin{theorem}
Si \(u_n \leq v_n, \forall n\) y \(v_n\) converge, luego \(u_n\) converge.
\end{theorem}
\begin{theorem}
Si \(u_n \geq v_n,  \forall n\) y \(v_n\) diverge, luego \(u_n\) diverge.
\end{theorem}
\begin{theorem}[Criterio de D'Alembert]
Sea \(s_1\) una serie con t\'erminos positivos. Y sea \[
\limi{n}{\infty}{\frac{u_{n+1}}{u_n}}{l} \]
entonces:\\
1) La serie converge cuando \(l < 1\)\\
2) La serie diverge cuando \(l > 1\)\\
(Cuando \(l = 1\) pueden darse los dos casos).
\end{theorem}
\begin{theorem}[Criterio de Cauchy]
Sea \(s_1\) una serie con t\'erminos positivos. Y sea \[
\limi{n}{\infty}{\sqrt[n]{u_n}}{l}\]
entonces:\\
1) La serie converge cuando \(l < 1\)\\
2) La serie diverge cuando \(l > 1\)\\
(Cuando \(l = 1\) pueden darse los dos casos).
\end{theorem}
\begin{theorem}[Criterio Integral]
Sea \(s_n\) una serie tal que
\[
s_1 \geq s_2 \geq s_3 ...
\]
Y sea \(f \) una funci\'on cont\'inua tal que \( f(i) = s_i \). Entonces se
cumple:\\
1) Si \( \int_1^{\infty} f(x) \ dx \) converge, tambi\'en converge \(s_n\)
\end{theorem}

\subsection{Teorema fundamental del c\'alculo}
Sea \(f\) integrable sobre \([a,b]\), luego su integral es derivable y
\[
\frac{d}{dx} \int_a^x f(t) \ dt = f(x)
\]

\[
\frac{d}{dx} \int_{a(x)}^{b(x)} \ dt = f(b(x))b'(x) - f(a(x))a'(x)
\]
\subsection{Regal de Barrow}
\subsection{M\'etodo de sustituci\'on}
\subsection{M\'etodo de integraci\'on por partes}
\subsection{M\'etodode integraci\'on por fraccinoes simples}

\end{document}
