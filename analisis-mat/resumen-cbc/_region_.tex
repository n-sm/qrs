\message{ !name(analisis-mat.tex)}\documentclass[12pt,a4paper]{article}
\usepackage[utf8]{inputenc}
\usepackage[spanish]{babel}
\usepackage{amsmath}
\usepackage{amsfonts}
\usepackage{amssymb}
\usepackage{amsthm}
\usepackage{parskip}
\usepackage[left=2cm,right=2cm,top=2cm,bottom=2cm]{geometry}
\newtheorem{mydef}{Definici\'on}[section]
\newtheorem{theorem}{Teorema}[section]
\newtheorem{ejs}{Ejemplo}[section]
\begin{document}

\message{ !name(analisis-mat.tex) !offset(-3) }


\section{Logaritmos}
Veamos la siguiente \emph{regla del cambio de base}:

Para todo \( x \)  se cumple \( x^{\log_x(M)} = M \). De esto se sigue
\( a^{\log_a(b)} = b \) y \( b^{\log_b(M)} = M \). Sustituyendo
obtenemos: \(M= \big(a^{\log_a b}\big)^{\log_b M} \) que es lo mismo que
\(M= a^{(\log_a b)(\log_b M)}   \). Con lo cual, tenemos:


\[a^{\log_a M} = a^{(\log_a b)(\log_b M)} \]

De los cual se sigue \( \log_a M = (\log_a b)(\log_b M) \)  y
finalmente:

\begin{equation}
\log_b(M) = \frac{\log_a(M)}{\log_a(b)}
\end{equation}


\section{L\'imites}

\subsection{Primero, consideremos las definiciones siguientes.}
\vspace{.5cm}


\begin{align*}
&\lim_{x \to a}f(x) = l  &  \equiv \qquad  &
( \forall \epsilon > 0 \ \exists \delta > 0 \ \forall x)  \qquad   0<|x-a|< \delta  \to  |f(x) - l| <
\epsilon  & \\
&\lim_{x \to \infty}f(x) = l  &  \equiv  \qquad &
(\forall \epsilon > 0 \ \exists N \ \forall x) \qquad  x > |N|  \to
 |f(x) - l| < \epsilon & \\
&\lim_{x \to a}f(x) = \infty  &  \equiv \qquad  &
(\forall M > 0 \ \exists \delta > 0 \ \forall x \neq a) \qquad   0<|x-a| < \delta  \to  |f(x)| >
M  & \\
&\lim_{x \to a}f(x) = 0  &  \equiv   \qquad  &
(\forall \epsilon > 0 \ \exists \delta > 0 \ \forall x) \qquad   0<|x-a| < \delta  \to  |f(x)| <
\epsilon & \\
&\lim_{x \to \infty}f(x) = 0  &  \equiv \qquad  &
(\forall \epsilon > 0 \ \exists N \ \forall x) \qquad   x > N  \to
 |f(x)| < \epsilon & \\
\end{align*}


\subsection{Algunos teoremas.}
%\vspace{1.5cm}
\begin{align*}
& \lim_{x \to a} k + f(x) & =  \qquad & k + \lim f(x)& \\
& \lim_{x \to 0} \frac{1}{x} & =   \qquad  & \infty & \\
& \lim_{x \to a}\Big(f(x) + g(x)\Big) & =   \qquad  &   \lim_{x \to a} f(x) + \lim_{x  \to a} g(x)&  \\
\end{align*}




\begin{theorem}

Si \( f(x) = c+g(x) \)  y  \(  \lim_{x \to  a}g(x)=0  \), luego \[ \lim_{x \to a} f(x) = c \].

\end{theorem}


\begin{theorem}
\[\text{Si}\qquad\lim_{x \to a}f(x)=0\qquad\text{y}\qquad \lim_{x \to a}g(x)\neq0\qquad\text{y}\qquad\lim_{x \to a}g(x)\neq\infty\]
\[\text{Luego:}\qquad\lim_{x \to a}\frac{f(x)}{g(x)}=0\]
\end{theorem}

\vspace{.5cm}

\subsection{M\'as definiciones.}
\begin{mydef}[Infinitesimales del mismo orden]
Sean $f(x)$ y $g(x)$ infinitesimales, cuando $x \to a$. Luego, si la
funci\'on:
\[\lim_{x \to a}\frac{f(x)}{g(x)}\]
tiene un l\'imite finito y distinto de cero, los infinitesimales
$f(x)$ y $g(x)$ son llamados \emph{del mismo orden}.
\end{mydef}
\begin{mydef}[Infinitesimales el orden superior e inferior]
Sean $f(x)$ y $g(x)$ infinitesimales, cuando $x \to a$. Luego, si la
funci\'on:
\[\lim_{x \to a}\frac{f(x)}{g(x)}=0\qquad\text{y}\qquad\lim_{x \to a}\frac{g(x)}{f(x)}=\infty\]
entonces, la infinitesimal $f(x)$ se llama \emph{infinitesimal de
  orden superior a} $g(x)$ y \'esta \'ultima se denomina
\emph{infinitesimal de orden inferior a} $f(x)$
\end{mydef}

\begin{mydef}[Infinitesimal de orden k respecto de otra]
Sean $f(x)$ y $g(x)$ infinitesimales, cuando $x \to a$. Luego, si la
funci\'on:
\[\lim_{x \to a}\frac{f(x)}{g(x)^k}=l\neq0\neq\infty\]
entonces, la infinitesimal $f(x)$ se llama \emph{infinitesimal de
  orden k respecto de} $g(x)$
\end{mydef}


\begin{mydef}[Infinitesimales equialentes]
Sean $f(x)$ y $g(x)$ infinitesimales, cuando $x \to a$. Luego, si la
funci\'on:
\[\lim_{x \to a}\frac{f(x)}{g(x)}=1\]
entonces, las infinitesimales $f(x)$ y $g(x)$ se llaman
\emph{equivalentes}.
\end{mydef}

\section{Continuidad}

\begin{mydef}[Continuidad de una funci\'on]

Una funci\'on \( f(x) \) es \emph{cont\'inua} en el punto \( x_0 \) si
est\'a definida en la vecindad de dicho punto y

\[ \lim_{\Delta x \to 0} \Delta y = 0 \]

O, lo que es igual:


\[ \lim_{\Delta x \to 0} (f(x_0+\Delta x)-f(x_0)) = 0 \]

Y como \( \lim(a+b) = \lim a + \lim b \), tambi\'en podemos escribir:

\[ \lim_{\Delta x \to 0} f(x_0+ \Delta x) = f(x_0) \]

Lo que es igual a:

\[ \lim_{x \to x_0} f(x) = f(x_0) \]



\end{mydef}

\section{Derivadas}

\begin{mydef}[Derivable en \( a \) ]
La funci\'on \( f \) es \emph{derivable en} \( a \)  si existe
\[ \lim_{h \to 0} \frac{f(a+h)-f(a)}{d} \] 
\end{mydef}

\begin{theorem}
Si f es derivable en a, entonces es cont\'inua en \( a \) .
\end{theorem}

\begin{theorem}
\[ (\log_a(x))' = \frac{1}{x}\log_a(e) \]
\end{theorem}
\begin{proof}

\[ \frac{\log_a(x+h)-\log_a(x)}{h} \quad = \quad
\log_a\Big(\frac{x+h}{x}\Big)\Big(\frac{1}{h}\Big) \quad = \quad
\log_a\Big(1+\frac{h}{x}\Big)\Big(\frac{1}{h}\Big) \quad = \quad
 \]


\[  \quad = \quad \log_a\Big(1+\frac{h}{x}\Big)\Big(\frac{x}{h}\Big)\Big(\frac{1}{x}\Big)
 \quad = \quad
 \Big(\frac{1}{x}\Big)\log_a\Big(1+\frac{h}{x}\Big)^{\Big(\frac{x}{h}\Big)} \]

Y como \( \lim_{h \to 0}(1+\frac{h}{x})^{\frac{x}{h}}  =  e \) se
deduce entones que 


\[ \Big(\log_a(x)\Big)' \quad = \quad \frac{1}{x} \log_a(e) \]
\end{proof}
Corolario: \( \Big( ln(x) \Big)' = \frac{1}{x}.  \) 
\begin{proof}
\( \ln(e)=1 \) 
\end{proof}

\subsection{Reglas de derivaci\'on}
En la siguiente tabla \( x \) reprsenta un variable, \( a \) y \( k \) 
constantes y \( u \) y \( v \) funciones de \( x \) .

{\renewcommand{\arraystretch}{1.2} 
\begin{tabular}{l l}
\( f \)  & \( f' \) \\
\hline \\
\( x^k \) & \( ku^{k-1} \)  \\
\( u^k \)  & \( ku^{k-1}u' \)  \\
\( a^x \)  & \( a^x\ln(a) \)  \\
\( a^u \)  & \( a^u(\ln a)u' \)  \\
\( u^v \)  & \( vu^{v-1}u'+u^v(\ln u)v' \)  \\
\( log_a(x) \) & \(  \big(\frac{1}{x}\big)\log_a(e) \) \\
\( log_a(u) \) & \(  \big(\frac{1}{u}\big)\log_a(e)u' \) \\
\( \frac{k}{x} \) & \( -\frac{k}{x^2} \) \\
\end{tabular}
}
\section{Sucesiones}

\begin{mydef}{Sucesi\'on infinita}

Una sucesi\'on infinita de n\'umeros reales es una funci\'on cuyo
dominio es \(\mathbb{N}\).
\end{mydef}

\begin{mydef}{Convergencia}

Una sucesi\'on \( \{ a_n \} \) \emph{converge hacia \( l \)} si  \(
(\forall \epsilon > 0) (\exists N) (\forall n)  :   n > N
\quad \to \quad | a_n - l | < \epsilon \) y se escribe 

\[ \lim_{n \to \infty} a_n = l \]


\end{mydef}


\begin{theorem}
Sea \( f \) una funci\'on definida en un intervalo abierto que contiene
\( c \), excepto quiz\'a en \( c \) mismo, con:

\[ \lim_{x \to c} f(x) = l \]

Y sup\'ongase adem\'as que \( \{ a_n \} \) es una sucesi\'on tal que:
\begin{enumerate}
\item cada \( a_n \) pertenece al dominio de \( f \),
\item cada \( a_n \neq c \),
\item \(\lim_{n \to \infty} a_n = c\)
\end{enumerate}

Entonces, la sucesi\'on \( \{ f(a_n) \} \) satisface

\[ \lim_{n \to \infty} f(a_n) = l\]

Rec\'iprocamente, si esto se cumple para toda sucesi\'on \( \{ a_n \}
\) que datisface las condiciones anteriores, entonces se cumple que

\[ \lim_{x \to c} f(x) = l \]


\end{theorem}

\begin{proof}
Para la primer parte de la prueba designemos las hip\'otesis as\'i:

A1.
\[ \forall \epsilon > 0 \ \exists \delta > 0 \ \forall x \quad
(0<|x-c|< \delta) \quad \to \quad (|f(x) - l| < \epsilon) \]

A2. 
\[ \forall \epsilon > 0 \ \exists N \ \forall n \quad  (n > N) \quad \to \quad (a_n - l| < \epsilon) \]

Seg\'un como se elija \( \epsilon \) en la segunda f\'ormula, tenemos
que 


\[ | f(a_n) - l | < \epsilon \]

(es decir, usamos \( \forall \epsilon > 0 \ \exists N \ \forall n
\quad  (n > N) \quad \to \quad (|a_n - l| < \epsilon) \)
que afirma el antecedente de A1).

Con esto hemos demostrado que

\[ \lim_{n \to \infty} f( a_n )= l \]

Rec\'iprocamente, designemos ahora ls hip\'otesis as\'i:

B1. 
\[ \forall \epsilon > 0 \ \exists N \ \forall n \quad  (n > N) \quad
\to \quad ( |f(a_n) - l| < \epsilon) \]

B2.
\[ \forall \epsilon > 0 \ \exists N \ \forall n \quad  (n > N) \quad
\to \quad ( |a_n - c| < \epsilon) \]

Ahora supongamos -a los fines de una reducci\'onn al absurdo- que no
se cumple \( \lim_{x \to c} f( x )= l \). Esto ser\'ia decir que:


\[ \exists \epsilon > 0 \ \forall \delta > 0 \ \exists x \quad
(0<|x-c|< \delta) \quad \wedge \quad (|f(x) - l| > \epsilon) \]

As\'i, para alg\'un \( \epsilon \) se cumplir\'ia que para todo \( n
\) (que es como decir, para todo \( 1/n \) existe un n\'umero \( x_n
\) tal que 

\[ \exists \epsilon > 0 \ \forall n \exists x_n \quad
(0<|x_n-c|< 1/n) \quad \wedge \quad (|f(x_n) - l| > \epsilon) \]

(es decir, hemos sustitu\'ido \( \delta \) por \( 1/n \) y \( x \) por
\( x_n \)).

Con esto, la sucesi\'on \( \{ x_n \} \) converge hacia \( c \) (usamos
ac\'a \( \epsilon = 1/n \) y \( N = n \), y aplicamos la defiici\'on
). Pero \( f( \{ a_n \} \) no converger\'ia a \( l \),
contradiciendo la hip\'otesis.
\end{proof}

En los ejemplos siguientes \( N \) se refiere al \( N \) de la definici\'on
de l\'imite.

\begin{ejs}{La sucesi\'on \( \{ \frac{1}{n}\} \)} es convergente a
  cero.

\( M \): cualquier natural que verifique la condici\'on \( \frac{1}{N}
< \epsilon\) o sea \( m > \frac{1}{\epsilon} \) . (propiedd arquimediana.

\end{ejs}
\hspace{1mm}
\begin{ejs}{La sucesi\'on \( \{ x^n \} \), con \( |x|< 1 \) converge a
  cero}

Escribimos \( |x| = \frac{1}{(1+p)} \) (para \( p \) conveniente).

Y para N tomamos cualquiera que cumpla con \( \frac{1}{pN} < \epsilon \).

\end{ejs}
\hspace{1mm}
\begin{ejs}{La sucesi\'on \( \sum_{k=0}^n x^n \) llamada \emph{serie
geom\'etrica} converge a \( \frac{1}{1-x} \)  }


\end{ejs}
\hspace{1mm}
\begin{ejs}[La sucesi\'on \( \{(-1)^n\} \) no es convergente.]


\end{ejs}
\hspace{1mm}
\begin{ejs}[\(\lim_{n \to \infty} \{ ^n\sqrt{a}  \} = 1\)]
\begin{proof}
Tenemos que encontrar un \( n \) que cumpla con 

\[ |a^{(\frac{1}{n})}| - 1 < \epsilon \].

Se dan dos casos.

\textbf{Caso 1}

\[ 0<a<1 \]

En este caso \( a^{\frac{1}{n}} < 1 \) para todo \( n \in \mathbf{N}
\). Por lo tanto, tenemos que


\[ |a^{\frac{1}{n}}-1| = 1 - a^{\frac{1}{n}} \]

De este modo, si


\[ 1 - a^{1/n} < \epsilon \]

tenemos


\[ a^{1/n} > 1 - \epsilon \]

y por esto:

\[ n > \frac{1 }{\log_a(1-\epsilon)} \]


\textbf{Caso 2}


\[ a > 1 \]

Aqu\'i tenemos que 


\[ | a^{1/n}-1| = a^{1/n}-1 \]

Tenemos que halla un \( n \) tal que

\[ a^{1/n}-1 < \epsilon \]

\[ a^{1/n} < \epsilon +1\]


Lo cual se sigue de \( \log_a(\epsilon+1)=1/x \) , ya que:


\[ n > x \to 1/n < 1/x \]

y


\[ 1/n < 1/x \to a^{1/n} < a^{1/x} \]

Como, por definici\'on de logaritmo


\[ a^{1/n} = \epsilon+1 \quad \equiv \quad \log_a(\epsilon+1) = \frac{1}{n} \]

y hemos supuesto la parte derecha, tenemos:


\[ a^{1/n} < \epsilon+1 \]
\end{proof}

\end{ejs}

\section{As\'intotas}

\subsection{As\'intota Vertical}
Se llama As\'intota Vertical de una rama de una curva \( y = f(x) \) , a la recta paralela al eje \( y \)  que hace que la rama de dicha funci\'on tienda a infinito. Si existe alguno de estos dos l\'imites:

\[ \lim_{x \to a^-} f(x) = \pm\infty \]


\[ \lim_{x \to a^+} f(x) = \pm\infty \]

a la recta \( x = a \)  se la denomina \emph{as\'intota vertical}.

Ejemplos: logaritmo neperiano, tangente
\subsection{As\'intota horizontal}
Se llama As\'intota Horizontal de una rama de una curva \(  y = f(x) \)  a la recta paralela al eje \( x \)  que hace que la rama de dicha funci\'on tienda a infinito. Si existe el l\'imite:
\[  \lim_{x \to \pm\infty} f(x)= a , \]
siendo \( a \)  un valor finito, la recta \( y = a \)  es una as\'intota horizontal.

Ejemplos: funci\'on exponencial, tangente hiperb\'olica
\subsection{As\'intota oblicua}
La recta de ecuaci\'on \( y = mx + b ,  (m \neq 0) \)  ser\'a una as\'intota oblicua si: 
\[ \lim_{x \to \pm\infty}[f(x)-(mx+b)] = 0. \]

Los valores de \( m \)  y de \( b \)  se calculan con las f\'ormulas:  

\[ m = \lim_{x \to \pm\infty}{f(x) \over x} \]

\[ b = \lim_{x \to \pm\infty}{f(x)-mx} \]


\end{document}

\message{ !name(analisis-mat.tex) !offset(-440) }
