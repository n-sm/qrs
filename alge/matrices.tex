\documentclass[alge.tex]{subfiles}
%\documentclass[12pt,a4paper]{extarticle}
\makeatother

\begin{document}
\maketitle
\section{Matrices}
\begin{teo}
  Sean \(A\), \(B\) y \(C\) matrices de \(m \times n\) y sea \(\alpha\)
  un escalar. Entonces:
  \begin{enumerate}
  \item \(A + \b{0} = A\)
  \item \(0A = \b{0}\)
  \item \(A + B = B + A\)
  \item \((A + B) + C = A + (B + C)\)
  \item \(\alpha(A+B)=\alpha A + \alpha B\)
  \item \(1A=A\)
  \end{enumerate}
\end{teo}

\begin{teo}
  Sean \(\b{a}\), \(\b{b}\) y \(\b{c}\) n-vectores y sea
  \(\alpha\) un escalar.
  \begin{enumerate}
  \item \( \b{a \cdot 0} = 0 \)
  \item \( \b{a \cdot b} = \b{b \cdot a}\)
  \item \( \b{a \cdot} ( \b{b} + \b{c}) = \b{a \cdot
      b} + \b{ a \cdot c}\)
  \item \((\alpha \b{a}) \cdot \b{b} = \alpha (\b{a}
    \cdot \b{b})\)
  \end{enumerate}
\end{teo}

\begin{teo}[Ley asociativa para multiplicaci\'on de matrices]\label{teoAsoc}
  Sea \(A_{n \times m}, B_{m \times p} \mbox{ y }  C_{p \times q}\)
  matrices. Entonces:

  \[ A(BC) = (AB)C\]

\end{teo}

\begin{teo}[Leyes distibutivas para la multiplicaci\'on de matrices]\label{teoDistM}
  Si todas las sumas y productos est\'an definidos, entonces:
  \[A(B+C)=AB+AC \]
  \[(A+B)C=AC+BC \]
\end{teo}
\fbox{
  \begin{minipage}{\mplen}
    \vspace{.25cm}
    Operaciones elementales entre filas.
    \begin{enumerate}
    \item Multiplicar (o dividir) una fila por un n\'umero distinto de
      cero.
    \item Sumar un m\'ultiplo de una fila a otra.
    \item Intercambiar dos filas.
    \end{enumerate}
    \vspace{.25cm}
  \end{minipage}
}
\vspace{.25cm}

\fbox{
  \begin{minipage}{\mplen}
    \vspace{.25cm}
    Notaci\'on:
    \begin{enumerate}
    \item \(M_i(c)\) indica: multiplicar la i-\'esima fila de una matriz por el n\'umero c.
    \item \(A_{i,j}(C)\) indica: multiplicar la i-\'esima fila opr c y
      sum\'arsela a la j.
    \item \(P_{i,j}\) indoca: permutar las filas \(i\) y \(j\).
    \end{enumerate}
    \vspace{.25cm}
  \end{minipage}
}

\begin{mydef}[Forma escalonada reducida]\label{defFormEsc}
  Una matriz est\'a en \emph{forma escalonada reducida} si se cumplen:
  \begin{enumerate}
  \item Todas las filas que consisten en \'unicamente ceros (si existen)
    aparecen en la parte de abajo de la matriz.
  \item El primer n\'umero distino de cero (empezando por la izquierda)
    en cualquier fila que no consista \'unicamente en ceros, es 1.
  \item Si dos filas sucesivas no consisten \'unicamente en ceros,
    entonces el primer 1 en la fila inferior est\'a m\'as a la derecha
    que el primer 1 de la fila superior.
  \item Cualquier columnna que contenga el primer 1 de una fila tendr\'a
    cero en los dem\'as lugares.
  \end{enumerate}
\end{mydef}

\begin{mydef}[Forma escalonada]
  Una matriz est\'a en \emph{forma escalonada} si se cumplen los \'items
  (1), (2) y (3) de la Definici\'on \ref{defFormEsc}.
\end{mydef}


\vspace{.25cm}
\fbox{
  \begin{minipage}{\mplen}
    \vspace{.25cm}
    M\'etodos para resolver sistemas de ecuaciones:\\
    \begin{enumerate}
    \item Eliminaci\'on de Gauss-Jordan:\\
      Reducir la matriz a la forma escalonada reducida.
    \item Eliminaci\'on Gaussiana:
      Reducir la matriz a la forma escalonada, despejar la \'ultima inc\'ognita y luego usarsustituci\'on hacia atr\'as para despejar las otras inc\'ognitas.
    \end{enumerate}
    \vspace{.25cm}
  \end{minipage}
}

\begin{mydef}[Sistema homog\'eneo de ecuaciones]
  Un sistema de ecuaciones se llama \emph{homog\'eneo} si todas las
  constates \(b_1, b_2, ..., b_m\) son cero, es decir:
  \[
  \begin{matrix}
    a_{11}x_1 & + & a_{12}x_2 & + & ... & + & a_{1n}x_x & = & 0\\
    a_{21}x_1 & + & a_{22}x_2 & + & ... & + & a_{2n}x_x & = & 0\\
    . &  & . & & & & . & & .\\
    . &  & . & & & & . & & .\\
    . &  & . & & & & . & & .\\
    a_{m1}x_1 & + & a_{m2}x_2 & + & ... & + & a_{mn}x_x & = & 0\\
  \end{matrix}  
  \]
  Lo cual puede expresarse tambi\'en:
  \[A\b{x}=\b{0}\]
  donde \(A\) es la matriz de coeficientes \(\b{x}\) el vector
  \((x_1, ..., x_n)\) y \(\b{0}\) el vector de \(m\) ceros.
\end{mydef}

\begin{mydef}[Sistema homg\'eneo asociado]
  Dado el sistema de ecuaciones \se{b}, se llama
  \emph{sistema homog\'eneo asociado} a \se{b} al sistema:
  \[A\b{x}=\b{0}\]
\end{mydef}

\begin{teo}
  Si \(n > m\), el sistema homog\'eneo de \(m\) ecuaciones con \(n\)
  inc\'ognitas tiene un n\'umero infinito de soluciones.
\end{teo}

\begin{teo}
  Sean \(\b{x_1}\) y \(\b{x_1}\) soluciones al sistema no
  homog\'eneo \se{b}. Entonces su diferencia
  \(\b{x_1}\ -\b{x_1}\) es una soluci\'on del sistema homo
  g\'eneo asociado \se{0}.
\end{teo}
\begin{proof}
  Por la ley distributiva (teorema \ref{teoDistM}), tenemos: 
  \[A(\b{x_1} - \b{x_2}) = A\b{x_1} - A\b{x_2}\]
  Pero como \(\b{x_1}\) y \(\b{x_1}\) son soluciones al
  sistema, tenemos:
  \[ A\b{x_1} - A\b{x_2} = \b{b} - \b{b} = \b{0}\]
\end{proof}

Corolario: Sean \(b{x}\) e \(b{y}\) dos soluciones del sistema no
homog\'eneo \se{b}. Entoces existe un vector \(\b{h}\) que es
soluci\'on del sistema homog\'eneo tal que:
\[ \b{y} = \b{x} + \b{h}\]

La prueba se obtiene facilmente para \(\b{h} = \b{y} - \b{x}\).

\vspace{.25cm}
\fbox{
  \begin{minipage}{\mplen}
    \vspace{.25cm}
    De este modo, para encontrar todas las soluciones del sistma no
    homog\'eneo \se{b}, es necesario encontrar s\'olo \emph{una} soluci\'on
    a \se{b} y todas las coluciones asociadas a \se{0}.
    \vspace{.25cm}
  \end{minipage}
  \vspace{.25cm}
}

\begin{mydef}[Matriz identidad] \label{defMatInv}
  La \emph{matriz identidad} de \(n \times n\) es ma matriz \(M_{n
    \times n}\) en la que las componentes de la \emph{diagonal
    principal} son todos \(1\) y \(0\) los restantes.
\end{mydef}

\begin{teo}
  Sea \(A\) una matriz cuadrada de \(n \times n\). Entonces:
  \[AI_n = I_nA = A\]
\end{teo}
\begin{proof}
  Sea \(c_{ij}\) el i-j\'esimo elemento de \(AI_n\), de modo que:
  \[c_{ij}= \sum_{k=1}^n a_{ik}b_{kj}\]
  donde \(a_{ij}\) son los elementos de \(A\) y \(b_{ij}\) son los de
  \(I_n\). Es evidente por la definici\'on \ref{defMatInv} que 
  \[c_{ij} = a_{ij}\]
  Lo mismo puede oservarse respecto de \(I_nA\).
\end{proof}

\begin{mydef}[Inversa de una matriz]\label{defInv}
  Sean \(A\) y \(B\) matrices de \(n \times n\). Si 
  \[AB = BA = I\]
  Entonces \(B\) es \emph{la inversa de} \(A\) y se escribe
  \(A^{-1}\). De este modo:
  \[ AA^{-1} = A^{-1}A = I\]
  Si \(A\) tiene inversa, entonce decimos que es \emph{invertible}.
\end{mydef}

\begin{teo} \label{teoMatInvUni}
  Si una matriz cuadrada es invertible, entonces su inversa es \'unica.
\end{teo}
\begin{proof}
  Debemos mostrar que si \(B\) y \(C\) son dos inversas de \(A\),
  entonces \(B = C\).\\
  Por hip\'otesis tenemos:
  \[(i) \ AB=BA=I\]
  y
  \[(ii) \ AC=CA=I\]
  Sustituyendo, usando (ii), tenemos:
  \[B(AC) = BI = B\]
  Usando (i):
  \[(BA)C = IC = C\]
  Por la ley asociativa (teorema \ref{teoAsoc}), tenemos:
  \[B(AC)=(BA)C\]
  Sustituyendo, tenemos pues:
  \[B=C\]
\end{proof}

\begin{teo}
  Sean \(A\) y \(B\) matrices invertibles de \(n\times n\). Entonces,
  \(AB\) es invertible y
  \[(AB)^{-1}=B^{-1}A^{-1}\]
\end{teo}

\begin{proof}
  Por la definici\'on \ref{defInv}, tenemos que \(AB\) es la inversa de
  \(B^{-1}A^{-1}\) (es decir \(B^{-1}A^{-1} = (AB)^{-1}\)) si y s\'olo
  si \((B^{-1}A^{-1})(AB)=(AB)(B^{-1}A^{-1}) = I\), que es lo que
  probaremos.

  Por la ley asociativa (teorema \ref{teoAsoc}) podemos probar:
  \[(\inv{B}\inv{A})(AB)=\inv{B}(\inv{A} A)B = \inv{B}IB=I\]
  as\'i como:
  \[(AB)(\inv{B}\inv{A})=A(B\inv{B})\inv{A}) AI\inv{A}=A\inv{A} = I\]
\end{proof}
Consid\'erese ahora el sistema de \(n \times n\) \se{b}. Supo\'ongase
que \(A\) es invertible, luego:
\begin{align*}
  \inv{A}A\b{x} &=\inv{A}\b{b}\\
  I\b{x} &=\inv{A}\b{b}\\
  \b{x} &=\inv{A}\b{b}\\
\end{align*}

\vspace{.25cm}
\fbox{
  \begin{minipage}{\mplen}
    \vspace{.25cm}
    De este modo, si \(A\) es invertible, el sistema \se{b} tiene la
    soluci\'on \'unica \(\b{x}=\inv{A}\b{b}\).
    \vspace{.25cm}
  \end{minipage}
}

\vspace{1cm}
\fbox{
  \begin{minipage}{\mplen}
    \vspace{.25cm}
    \textbf{Procedimiento para calcular la inversa de una matriz cuadrada A}
    \begin{enumerate}
    \item Escribir la matriz aumentada \((A|I)\).
    \item Utilizar redicci\'on por renglones para reducir la matriz A a su
      forma escalonada.
    \item Decidir si la matriz A es invertible.
      \begin{enumerate}
      \item Si \(A\) puede ser reducida a la matriz identidad \(I\),
        entonces \(A\) es la matriz a la derecha de la barra vertical.
      \item Si la reducci\'on por renglones conduce a alg\'un
        rengl\'on de ceros a la izquierda de la barra vertical, la
        matriz \(A\) no es invertible.
      \end{enumerate}
    \end{enumerate}
    \vspace{.25cm}
  \end{minipage}
}

\vspace{1cm}
Observaci\'on: podemos expresar (a) y (b) diciendo que una matriz
cuadrada \(A\) es invertible si y s\'olo si su forma escalonada,
reducida por renglones, es la matriz identidad.
\end{document}