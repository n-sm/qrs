\documentclass[12pt,a4paper]{extarticle}
\usepackage[utf8]{inputenc}
\usepackage[spanish]{babel}
\usepackage{amsmath}
\usepackage{amsfonts}
\usepackage{amssymb}
\usepackage{amsthm}
\usepackage{float}
\usepackage{varioref}
\usepackage{currfile}
\title{\'Algebra}
\makeatother
%\usepackage{mdframed}
\usepackage[left=1.6cm,right=1.6cm,top=1.6cm,bottom=1.6cm]{geometry}
\newtheorem{teo}{Teorema}[section]
%\newmdtheoremenv{theorem}{Teorema}[section]
\newtheorem{ejs}{Ejemplo}[section]
\newtheorem{mydef}{Definici\'on}[section]
\newtheorem{corol}{Corolario}[teo]
%\newmdtheoremenv{mydef}{Definici\'on}[section]
\begin{document}
\maketitle
\section{Matrices}
\begin{teo}
Sean \(A\), \(B\) y \(C\) matrices de \(m \times n\) y sea \(\alpha\)
un escalar. Entonces:
\begin{enumerate}
\item \(A + \mathbf{0} = A\)
\item \(0A = \mathbf{0}\)
\item \(A + B = B + A\)
\item \((A + B) + C = A + (B + C)\)
\item \(\alpha(A+B)=\alpha A + \alpha B\)
\item \(1A=A\)
\end{enumerate}
\end{teo}

\begin{teo}
Sean \(\mathbf{a}\), \(\mathbf{b}\) y \(\mathbf{c}\) n-vectores y sea
\(\alpha\) un escalar.
\begin{enumerate}
\item \( \mathbf{a \cdot 0} = 0 \)
\item \( \mathbf{a \cdot b} = \mathbf{b \cdot a}\)
\item \( \mathbf{a \cdot} ( \mathbf{b} + \mathbf{c}) = \mathbf{a \cdot
    b} + \mathbf{ a \cdot c}\)
\item \((\alpha \mathbf{a}) \cdot \mathbf{b} = \alpha (\mathbf{a}
  \cdot \mathbf{b})\)
\end{enumerate}
\end{teo}

\begin{teo}[Ley asociativa para multiplicaci\'on de matrices]
Sea \(A_{n \times m}, B_{m \times p} \mbox{ y }  C_{p \times q}\)
matrices. Entonces:

\[ A(BC) = (AB)C\]

\end{teo}

\begin{teo}[Leyes distibutivas para la multiplicaci\'on de matrices]
Si todas las sumas y productos est\'an definidos, entonces:
\[A(B+C)=AB+AC \]
\[(A+B)C=AC+BC \]
\end{teo}
\fbox{
\begin{minipage}{15cm}
\vspace{.25cm}
Operaciones elementales entre filas.
\begin{enumerate}
\item Multiplicar (o dividir) una fila por un n\'umero distinto de
  cero.
\item Sumar un m\'ultiplo de una fila a otra.
\item Intercambiar dos filas.
\end{enumerate}
\vspace{.25cm}
\end{minipage}
}
\vspace{.25cm}

\fbox{
\begin{minipage}{15cm}
\vspace{.25cm}
Notaci\'on:
\begin{enumerate}
\item \(M_i(c)\) indica: multiplicar la i-\'esima fila de una matriz por el n\'umero c.
\item \(A_{i,j}(C)\) indica: multiplicar la i-\'esima fila opr c y
  sum\'arsela a la j.
\item \(P_{i,j}\) indoca: permutar las filas \(i\) y \(j\).
\end{enumerate}
\vspace{.25cm}
\end{minipage}
}

\begin{mydef}[Forma escalonada reducida]
Una matriz est\'a en forma escalonada reducida si se cumplen:
\begin{enumerate}
\item Todas las filas que consisten en \'unicamente ceros (si existen)
  aparecen en la parte de abajo de la matriz.
\item El primer n\'umero distino de cero (empezando por la izquierda)
  en cualquier fila que no consista \'unicamente en ceros, es 1.
\item Si dos filas sucesivas no consisten \'unicamente en ceros,
  entonces el primer 1 en la fila inferior est\'a m\'as a la derecha
  que el primer 1 de la fila superior.
\item Cualquier columnna que contenga el primer 1 de una fila tendr\'a
  cero en los dem\'as lugares.
\end{enumerate}
\end{mydef}

\begin{mydef}
\end{mydef}

\end{document}