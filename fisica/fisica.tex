\documentclass[12pt,a4paper]{article}
\usepackage[utf8]{inputenc}
\usepackage[spanish]{babel}
\usepackage{amsmath}
\usepackage{amsfonts}
\usepackage{amssymb}
\usepackage[left=2cm,right=2cm,top=2cm,bottom=2cm]{geometry}
%\pagenumbering{gobble}
\newtheorem{mydef}{Definici\'on}[section]
\newtheorem{theorem}{Teorema}[section]
\setlength{\parindent}{0pt}
\begin{document}

\section{Cinem\'atica}

\begin{mydef}[Vector posici\'on]
Este vector parte del or\'igen de coordenadas y llega al punto que
se quiere identificar.

Sus componente son las coordenadas de dicho punto. Se escribe as\'i:

\[ \vec{r}= r_x \hat{x} + r_y \hat{y} + r_z \hat{z} \]

\end{mydef}
\begin{mydef}[Trayectoria]
Es la curva que recorri\'o el m\'ovil al desplazarse.
\end{mydef}

\begin{mydef}[Vector desplazamiento]
Si una part\'icula se mueve del punto \(P_1\) al \(P_2\) llamamos
\emph{vector desplazamiento} al que tiene origen en \(P_1\) y extremo
en \(P_2\).

El vector desplazamiento es independiente del recorrido de la
part\'icula.

Se calcula restanto al vector posici\'on de \(P_2\) el de \(P_1\), es
decir:
\[ \Delta \vec{r} = \vec{r}_2 - \vec{r}_1 \]
\end{mydef}
\begin{mydef}[Lapso o intervalo temporal]
Es el tiempo transcurrido  entre los instantes en que el vector
posici\'on es \(\vec{r}_1\) y  \( \vec{r}_2\) respectivamente. O sea:
\[\Delta t = t_2 - t_1\]
\end{mydef}
\begin{mydef}[Vector velocidad media]
\[ \vec{v}_m = \frac{\Delta \vec{r}}{\Delta t } \]
La direcci\'on y el sentido de este vector son las del desplazamiento.
\end{mydef}
\begin{mydef}[Vector velocidad instant\'anea]
Este vector tiene siempre la direcci\'on de la recta tangente a la
trayectoria.
El sentido del vector velocidad instant\'anea es el mismo que el del
movimiento.
\[ \vec{v} = \lim_{\Delta t \to 0} \frac{\Delta \vec{r}}{\Delta t} \]
\end{mydef}
\begin{mydef}[Vector aceleraci\'on media]
\[ \vec{a}_m =  \frac{\Delta \vec{v}}{\Delta t} = \frac{\vec{v}_2 -
  \vec{v}_1}{t_2 - t_1} \]
\end{mydef}
\begin{mydef}[Vector aceleraci\'on instant\'anea]
\[ \vec{a} = \lim_{\Delta t \to 0} \frac{\Delta \vec{v}}{\Delta t} 
 \]
\end{mydef}

\subsection{Movimiento Rectilineo Uniforme}
Es un movimiento rectil\'ineo con velocidad constante. O, lo que es lo
mismo, movimiento con aceleraci\'on nula. Queda pues caracterizado por:

\[
\vec{a}(t) = 0 \]
\[
\vec{v}(t) = \vec{v}_0 \]



\begin{mydef}[Ecuaci\'on horaria]
\[
x(t) = x_0 + v_0(t - t_0)\]
La posici\'on en en \(x\) el tiempo \(t\) es igual a la posici\'on inicial
\(x_0\) m\'as el producto de la velocidad por el lapso temporal. Como la
velocidad es constante basta tomas \(v_0\).
\end{mydef}

\subsection{Movimiento Rectil\'ineo Uniformemente Variado}
En este caso la aceleraci\'on media es constante. De este modo, la
funci\'on de la velocidad en funci\'on del tiempo es:

\[ v(t) = v_0 + a(t-t) \]

La \textbf{\textit{ecuaci\'on horaria}} es:

\[
x(t) = x_0 + v_0(t-t_0) + \frac{1}{2} a (t-t_0)^2\]

\subsubsection{Caida libre y Tiro vertical}
Son casos particulares del \textbf{MRUV} con \( a = |\vec{g}| \) y
\(a=- |\vec{g}|\) seg\'un el caso y el marco de referencia.

\subsection{Movimiento relativo}

\begin{mydef}[Traslaci\'on uniforme]
Es el que se da cuando hay dados dos sistemas de referencia
\textbf{A} y \textbf{B}, uno de los cuales se mueve con respecto al otro
 de modo que su or\'igen hace un \textbf{MRU} y sus ejes mantienen
 siempre la misma orientaci\'on.

Todos los puntos de \textbf{B} se mueven con respecto a \textbf{A} con
velocidad:
\[ \vec{v}_{BA}\]
que es la velocidad de \textbf{B} con respecto a \textbf{A}. Se cumple
que:

\[ \vec{v}_{AB} = -\vec{v}_{BA}\]
\end{mydef}

Supongamos que hay ubna partícula ubicada en el punto \(P\) cuya
posición es \(\vec{r}_{PB}\)

\begin{mydef}[Sistemas de refererencia inerciales]
Se denominan \emph{sistemas de referencia inerciales} aquellos que
est\'an en reposo o bien se mueven con velocidad rectil\'inea y
uiforme.

El estado de reposo asoluto no existe (o no tiene sentido hablar de
\'el)

Adem\'as: 
\begin{itemize}
\item Las velocidades de los objetos son magnitudes relativas, cuyo
  valor depende del sistema de referencia inercial en el que se midan
  (medidas desde sistemas de referencai inerciales, las velocidades
  son relativas).
\item Dos sistemas de referencia inerciales obtendr\'an la misma
  aceleraci\'on para un objeto externo (o sea, medidas desde sistemas
  inerciales, las aceleraciones son absolutas, no relativas).
\end{itemize}
\end{mydef}

\section{Din\'amica}
\begin{mydef}[Peso]
El peso de un objeto de masa \(m\) es la fuerza con que la Tierra lo
atrae, y vale \[\vec{P} = m\vec{g}\]
 La aceleraci\'on de la gravedad est\'a siempre dirigida hacia el
 centro de la tierra.
\end{mydef}
\begin{mydef}[Inercia]
La inercia es la dificultad que presenta un objeto para moerlo o para
deternerlo.
\end{mydef}
\begin{mydef}[Fuerza]
Una fuerza es cualquier acci\'on, ejercida desde el exterior de un
cuerpo, que modifica su velocidad o estado de reposo.
\end{mydef}
\begin{theorem}
[Ley de inercia o Primer Ley de Newton]
Todo cuerpo contin\'ua en su estado de reposo o movimiento uniforme y
en l\'inea recta, si sobre su cuerpo no act\'ua ninguna fuerza.

De otro modo, esta ley puede interpretarse diciendo que:\\
Una fuerza es cualquier acci\'on que permite alterar la velocidad de
una part\'icula.
\end{theorem}
\begin{theorem}[Segunda Ley de Newton]
Si sobre una part\'icula de masa \(m\) act\'ua una fuerza \(F\),
\'esta le comunica una aceleraci\'on que viene dada por el cociente de
la fuerza por la masa.
\[\vec{a}=\frac{\vec{F}}{m}\]
O sea \[\vec{F}=m\vec{a}\], lo quem descompuesto quivale a:
\[F_x=ma_x,\quad  F_y=ma_y, \quad F_z=ma_z\]
\end{theorem}

\begin{mydef}[Principio de superposici\'on de fuerzas]
Este principio afirma que el efecto de un conjunto de fuerzas es el
mismo que ejerce la fuerza resultante.
Por ejemplo, si tenemos dos fuerzas \(\vec{F_1}\) y \(\vec{F_2}\),
aplicadas sobre un objeto, la aceleraci\'on resultante es la suma de
las aceleraciones que le provocar\'ia cada fuerza por separado:
\[\vec{a}=\frac{\vec{F_1}+\vec{F_2}}{m}=\frac{\vec{F_1}}{m}+\frac{\vec{F_2}}{m}= \vec{a_1}+\vec{a_2}\]
\end{mydef}

\begin{theorem}[Tercera Ley de Newton (principio de acci\'on y reacci\'on)]
Las fuerzas siempre se ejercen por pares: si un cuerpo \(A\) ejerce
una fuerza \(\vec{F}_{A \to B}\) sobre un cuerpo \(B\), entonces el
cuerpo \(B\) ejerce una fuerza igual y contraria sobre el cuerpo
\(A\), \(\vec{F}_{b \to A}\).
\[\vec{F}_{B \to A} = -\vec{F}_{A \to B}\]

\end{theorem}

\end{document}
