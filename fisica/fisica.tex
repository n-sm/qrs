\documentclass[12pt,a4paper]{article}
\usepackage[utf8]{inputenc}
\usepackage[spanish]{babel}
\usepackage{amsmath}
\usepackage{amsfonts}
\usepackage{amssymb}
\usepackage[left=2cm,right=2cm,top=2cm,bottom=2cm]{geometry}
\pagenumbering{gobble}
\newtheorem{mydef}{Definici\'on}[section]

\setlength{\parindent}{0pt}
\begin{document}

\section{Cap. 1}

\begin{mydef}[Vector posici\'on]
Este vector parte del or\'igen de coordenadasy llega al punto que
se quiere identificar.

Sus componente son las coordenadas de dicho punto. Se escribe as\'i:

\[ \vec{r}= r_x \hat{x} + r_y \hat{y} + r_z \hat{z} \]

\end{mydef}
\begin{mydef}[Trayectoria]
Es la curva que recorri\'o el m\'ovil al desplazarse.
\end{mydef}

\begin{mydef}[Vector desplazamiento]
Si una part\'icula se mueve del punto \(P_1\) al \(P_2\) llamamos
\emph{vector desplazamiento} al que tiene origen en \(P_1\) y extremo
en \(P_2\).

El vector desplazamiento es independiente del recorrido de la
part\'icula.

Se calcula restanto al vector posici\'on de \(P_2\) el de \(P_1\), es
decir:
\[ \Delta \vec{r} = \vec{r}_2 - \vec{r}_1 \]
\end{mydef}
\begin{mydef}[Lapso o intervalo temporal]
Es el tiempo transcurrido  entre los instantes en que el vector
posici\'on es \(\vec{r}_1\) y  \( \vec{r}_2\) respectivamente. O sea:
\[\Delta t = t_2 - t_1\]
\end{mydef}
\begin{mydef}[Vector velocidad media]
\[ \vec{v}_m = \frac{\Delta \vec{r}}{\Delta t } \]
La direcci\'on y el sentido de este vector son las del desplazamiento.
\end{mydef}
\begin{mydef}[Vector velocidad instant\'anea]
Este vector tiene siempre la direcci\'on de la recta tangente a la
trayectoria.
El sentido del vector velocidad instant\'anea es el mismo que el del
movimiento.
\[ \vec{v} = \lim_{\Delta t \to 0} \frac{\Delta \vec{r}}{\Delta t} \]
\end{mydef}
\begin{mydef}[Vector aceleraci\'on media]
\[ \vec{a}_m =  \frac{\Delta \vec{v}}{\Delta t} = \frac{\vec{v}_2 -
  \vec{v}_1}{t_2 - t_1} \]
\end{mydef}
\begin{mydef}[Vector aceleraci\'on instant\'anea]
\[ \vec{a} = \lim_{\Delta t \to 0} \frac{\Delta \vec{v}}{\Delta t} 
 \]
\end{mydef}

\subsection{Movimiento Rectilineo Uniforme}
Es un movimiento rectil\'inea con velocidad constante. O, lo que es lo
mismo, movimiento co naceleraci\'on nula. Queda pues caracterizado por:

\[
\vec{a}(t) = 0 \]
\[
\vec{v}(t) = \vec{v}_0 \]



\begin{mydef}[Ecuaci\'on horaria]
\[
x(t) = x_0 + v_0(t - t_0)\]
La posici\'on en en \(x\) el tiempo \(t\) es igual a la posici\'on inicial
\(x_0\) m\'as el producto de la velocidad por el lapso temporal. Como la
velocidad es constante basta tomas \(v_0\).
\end{mydef}

\subsection{Movimiento Rectil\'ineo Uniformemente Variado}
En este caso la aceleraci\'on media es constante. De este modo, la
funci\'on de la velocidad en funci\'on del tiempo es:

\[ v(t) = v_0 + a(t-t) \]

La \textbf{\textit{ecuaci\'on horaria}} es:

\[
x(t) = x_0 + v_0(t-t_0) + \frac{1}{2} a (t-t_0)^2\]

\subsubsection{Caida libre y Tiro vertical}
Son casos particulares del \textbf{MRUV} con \( a = |\vec{g}| \) y
\(a=- |\vec{g}|\) seg\'un el caso y el marco de referencia.

\subsection{Movimiento relativo}

\begin{mydef}[Traslaci\'on uniforme]
Es el que se da cuando hay dados dos sistemas de referencia
\textbf{A} y \textbf{B}, uno de los cuales se mueve con respecto al otro
 de modo que su or\'igen hace un \textbf{MRU} y sus ejes mantienen
 siempre la misma orientaci\'on.

Todos los puntos de \textbf{B} se mueven con respecto a \textbf{A} con
velocidad:
\[ \vec{v}_{BA}\]
que es la velocidad de \textbf{B} con respecto a \textbf{A}. Se cumple
que:

\[ \vec{v}_{AB} = -\vec{v}_{BA}\]
\end{mydef}

Supongamos que hay ubna partícula ubicada en el punto \(P\) cuya
posición es \(\vec{r}_{PB}\)
\end{document}
