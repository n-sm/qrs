\documentclass[12pt,a4paper]{article}
\usepackage[utf8]{inputenc}
\usepackage[spanish]{babel}
\usepackage{amsmath}
\usepackage{amsfonts}
\usepackage{amssymb}
\usepackage[left=2cm,right=2cm,top=2cm,bottom=2cm]{geometry}
\pagenumbering{gobble}

\title{R}

\begin{document}
\maketitle

\section{Instalaci\'on y obtenci\'on de ayuda}
La instalaci\'on de paquetes se realiza mediante el siguiente comando:
\begin{verbatim}
> install.packages(``nombreDelPaquete'')
\end{verbatim}
Tras lo cual se despliega un men\'u para elegir el mirror.
Para usar el paquete, luego de su instalaci\'onm escribimos:
\begin{verbatim}
library(``nombreDelPaquete'')
\end{verbatim}

\section{Importar y exportar datos}
Para archivos de excel (.xls) usamos el paquete \texttt{RODBC}.
Supongamos que en nuestro directorio de trabajo tenemos un archivo
de nombre \texttt{personalidad.xls}, que es un documento de Excel, y
que tiene una hoja llamada \texttt{16pf} donde hay una tabla de \(n\)
por \(m\) datos, con \(n\) sujetos y los \(m\) \'items del test 16pf de
Cattell (\(185\)). Entonces, para acceder a dicha base desde
\texttt{R} escribimos:
\begin{verbatim}
> library("RODBC")
> cnct <- odbcConnectExcel("personalidad.xls")
> datos <- sqlQuery(cnct, "select * from \"16pf$\"")
\end{verbatim}
\end{document}
